\subsection{Subobject}

\prob{3.1.7.} Since $S\subseteq R$, consider $1_S = 1_S\cdot 1_S$ in $R$. Then $1_S = 1_S\cdot 1_R = 1_S\cdot 1_S$, and thus $1_S\cdot (1_R-1_S) = 0$. Now consider the equation in $R$. Because $R$ is an integral domain, $1_R - 1_S = 0$, which gives $1_R = 1_S$.\\
\\
\prob{3.1.14.} Let $T = \ds \bigcap_{S\subseteq Y \leq X} Y$. (Existence) Trivially, $S\subseteq T$, and $T\leq X$ by 3.1.10. Also, if $S\subseteq S'\leq X$ then $T\subseteq S'$, and again by 3.1.10, $T\leq S'$. Thus $T$ is a minimal sub-$\square\square$ of $X$ generated by $S$. (Uniqueness) Let $U, V$ both be a minimal sub-$\square\square$ of $X$ generated by $S$. By definition, $S\subseteq U\leq X$ and $S\subseteq V\leq X$. Since $S\subseteq V$, $U\leq V$ by the minimality of $U$. Similarly, $V\leq U$. Thus $U = V$ and $T=\span{S}$ is unique.\\
\\
\prob{3.1.16} [Subgroup of abelian group] = [$\Z$-submodule of $\Z$-module]
\begin{enumerate}
	\item (\mimp) For abelian groups $A, B$, suppose $B\leq A$. Then the submodule criterion can be re-written as the following statements.
	\begin{enumerate}
		\item ...
	\end{enumerate}
\end{enumerate}
\pagebreak