\section{Algebraic Structures I}
\subsection{Algebraic Structure}
\textbf{대수학}은 \textbf{algebraic structure}(\textbf{대수적 구조})를 공부하는 학문이다. 대수적 구조란 어떤 집합에 몇 개의 연산 구조가 주어진 것을 뜻한다. 우리는 \textbf{associative} binary operation 만을 생각한다.\footnote{모든 binary operation 은 associative 라고 가정한다.}\\
\\
\defn{1.1.1.} 이항연산 $\ast$ 를 갖는 집합 $G$ 가 다음 조건들
\begin{enumerate}
	\item[\sffamily (G1)] [모든 $g\in G$ 에 대하여 $g\ast e=e\ast g = g$] 인 원소 $e\in G$ 가 존재.
	\item[\sffamily (G2)] 각 $g \in G$ 에 대하여 [$g\ast \tilde{g} = \tilde{g}\ast g = e$ 인 원소 $\tilde{g}\in G$ 가 존재].
\end{enumerate}
을 만족하면 $(G, \ast)$ 를 \textbf{group}(\textbf{군}) 이라고 한다.\\

어떤 추가 조건들이 주어져 있는가에 따라 대수적 구조의 이름이 달라진다. 만약 위 정의에서 이항연산을 가진 집합 $G$ 가
\begin{enumerate}
	\item 아무런 추가 조건도 갖지 않으면 $G$ 를 \textbf{semigroup} 이라 한다.
	\item {\sffamily (G1)} 만을 만족하면, $G$ 를 \textbf{monoid} 라 한다.
	\item Group $G$ 가 $\abs{G}<\infty$ 도 만족하면, $G$ 를 \textbf{finite group} 이라 한다.
	\item Group $G$ 가 [$g\ast h = h\ast g$ for all $g, h\in G$] 도 만족하면 $G$ 를 \textbf{commutative group} (또는 \textbf{abelian group}) 이라 한다.
\end{enumerate}~
\\
\prob{1.1.3.} Consider $\N \cup \{0\}$ with addition. Associativity holds trivially, and $e = 0$. But there is no inverse for elements in $\N$.\\
\\
\defn{1.1.4.} 