\section{Algebraic Structures I}
\subsection{Algebraic Structure}
\textbf{대수학}은 \textbf{algebraic structure}(\textbf{대수적 구조})를 공부하는 학문이다. 대수적 구조란 어떤 집합에 몇 개의 연산 구조가 주어진 것을 뜻한다. 우리는 \textbf{associative} binary operation 만을 생각한다.\footnote{모든 binary operation 은 associative 라고 가정한다.}\\
\\
\defn{1.1.1.} 이항연산 $\ast$ 를 갖는 집합 $G$ 가 다음 조건들
\begin{enumerate}
	\item[\sffamily (G1)] [모든 $g\in G$ 에 대하여 $g\ast e=e\ast g = g$] 인 원소 $e\in G$ 가 존재.
	\item[\sffamily (G2)] 각 $g \in G$ 에 대하여 [$g\ast \tilde{g} = \tilde{g}\ast g = e$ 인 원소 $\tilde{g}\in G$ 가 존재].
\end{enumerate}
을 만족하면 $(G, \ast)$ 를 \textbf{group}(\textbf{군}) 이라고 한다.\\

어떤 추가 조건들이 주어져 있는가에 따라 대수적 구조의 이름이 달라진다. 만약 위 정의에서 이항연산을 가진 집합 $G$ 가
\begin{enumerate}
	\item 아무런 추가 조건도 갖지 않으면 $G$ 를 \textbf{semigroup} 이라 한다.
	\item {\sffamily (G1)} 만을 만족하면, $G$ 를 \textbf{monoid} 라 한다.
	\item Group $G$ 가 $\abs{G}<\infty$ 도 만족하면, $G$ 를 \textbf{finite group} 이라 한다.
	\item Group $G$ 가 [$g\ast h = h\ast g$ for all $g, h\in G$] 도 만족하면 $G$ 를 \textbf{commutative group} (또는 \textbf{abelian group}) 이라 한다.
\end{enumerate}~
\\
\prob{1.1.3.} Consider $\N \cup \{0\}$ with addition. Associativity holds trivially, and $e = 0$. But there is no inverse for elements in $\N$.\\
\\
\defn{1.1.4.} $R$ 과 $X$ 가 집합일 때, 함수 $R\times X \ra X$ 를 $X$-위의 $R$-\textbf{상수곱} (\textbf{scalar multiplication}) 이라 한다. $a\in R$ 과 $x\in X$ 의 상수곱은 $a\cdot x = ax$ 로 표기한다. 또, $R$ 의 원소는 \textbf{scalar} 라고 부른다.\\
\\
\defn{1.1.7.} (\textbf{\sffamily Ring}) 집합 $R$ 이 \textbf{덧셈}과 \textbf{곱셈}이라는 이름의 두 개의 이항연산을 갖고 있을 때 다음 조건
\begin{enumerate}
	\item[\sffamily (R1)] $(a+b)c = ac+bc$, $a(b+c)=ab+ac$ \quad ($a, b, c\in R$) \quad (\textbf{분배법칙}(\textbf{distributive law}))
	\item[\sffamily (R2)] $(R, +)$ 는 abelian group
\end{enumerate}
을 만족하면, $(R, +, \gop)$ 을 \textbf{ring}(\textbf{환}) 이라 한다.\\
\\
\prob{1.1.8.} \textbf{Typical examples of ring}
\begin{enumerate}
	\item $(\Z, +,\gop)$
	\item $(\R, +,\gop)$
	\item $(\mf{M}_{n, n}(\R), +, \gop)$
	\item $(\R[t], +, \gop)$
\end{enumerate}
For all 4 examples, each set is an abelian group under addition, and the distributive law holds.\\

Alway remember: \textbf{우리가 아는 것은 행렬 뿐이다. 어떤 수학적 object 를 만나더라도 우리는 행렬(벡터공간)부터 생각한다.}\\
\\
\defn{1.1.10.} (\textbf{\sffamily $R$-module}) $R = (R, +, \gop)$ 이 ring 이고, 집합 $M$ 이 이항연산 \textbf{덧셈}과 $R$-\textbf{상수곱}을 갖고 있다고 하자.\footnote{$M$ 의 덧셈과 $R$ 의 덧셈은 분명히 구별해야 한다.} 다음 조건
\begin{enumerate}
	\item[\sffamily (M1)] $r(x+y) = rx + ry$ \quad ($r\in R,\;\; x, y, \in M$)
	\item[\sffamily (M2)] $(r+s)x = rx+sx$ \quad ($r, s\in R, \;\; x\in M$)
	\item[\sffamily (M3)] $r(sx) = (rs)x$ \quad ($r, s\in R, \;\; x\in M$)
	\item[\sffamily (M4)] $(M, +)$ 는 abelian group
\end{enumerate}
을 만족하면, $(M, +, \sang)$ 을 $R$-\textbf{module} ($R$-\textbf{가군}) 이라 한다.\\
\\
\prob{1.1.11.} \textbf{Typical examples of module}
\begin{enumerate}
	\item $x = (x_1, \dots, x_n), y = (y_1, \dots, y_n) \in \R^n$, $r, s\in \R$.
	\begin{enumerate}
		\item $r(x+y) = r(x_1+y_1, \dots, x_n+y_n) = (rx_1+ry_1, \dots, rx_n+ry_n) = rx+ry$
		\item $(r+s)x = (rx_1+sx_1, \dots, rx_n+sx_n) = rx + sx $
		\item $r(sx) = r(sx_1, \dots, sx_n) = (rsx_1, \dots, rsx_n) = (rs)x$
		\item Vector space $\R^n$ is an abelian group under addition.
	\end{enumerate}
	\item The proof is identical to (1).
\end{enumerate}~
\\
\prob{1.1.12.} $\R$-vector space was defined on a field ($\R$), while $R$-module is defined on a ring. A vector space over a field is a module over that field.\\
\\
\defn{1.1.13.} (\textbf{$R$-algebra}) $R=(R, +, \gop)$ 이 ring 이고, 집합 $\mc{A}$ 가 이항연산 \textbf{덧셈}과 \textbf{곱셈}, 그리고 $R$-\textbf{상수곱}을 갖고 있다고 하자. 다음 조건
\begin{enumerate}
	\item[\sffamily (A1)] $(\mc{A}, +, \gop)$ 은 ring
	\item[\sffamily (A2)] $(\mc{A}, +, \sang)$ 은 $R$-module
	\item[\sffamily (A3)] $(ra)b = r(ab) = a(rb)$ \quad ($r\in R, \;\; a, b, \in\mc{A}$)
\end{enumerate}
을 만족하면, $(\mc{A}, +, \gop, \sang)$ 을 $R$-\textbf{algebra} ($R$-\textbf{대수}) 이라 한다.\\
\\
\prob{1.1.15.} \textbf{Typical examples of algebra}
\begin{enumerate}
	\item $\R[t]$ (the $\R$-\textbf{algebra of polynomials}, the \textbf{polynomial algebra} over $\R$)
	\begin{enumerate}
		\item \textit{Is $(\R[t], +, \gop)$ a ring?} Yes.
		\item \textit{Is $(\R[t], +, \sang)$ an $R$-module?} Yes.
		\item For $r\in \R$, $f(t), g(t)\in \R[t]$,
		$(rf(t))g(t) = r(f(t)g(t)) = f(t)(rg(t))$.
	\end{enumerate}
	\item $\mf{M}_{n, n}(\R)$ (the $\R$-\textbf{algebra of $(n\times n)$-matrices}, the \textbf{matrix algebra} over $\R$)
	\begin{enumerate}
		\item \textit{Is $(\mf{M}_{n, n}(\R), +, \gop)$ a ring?} Yes.
		\item \textit{Is $(\mf{M}_{n, n}(\R), +, \sang)$ an $R$-module?} Yes.
		\item For $r \in \R$, $A, B \in \mf{M}_{n, n}(\R)$, $(rA)B = r(AB) = A(rB)$.
	\end{enumerate}
\end{enumerate}~
\\
\prob{1.1.17.} It is sufficient to only check {\sffamily (R1)}. For $a, b, c\in A$, $(a+b)c = 0 = 0 + 0 = ac + bc$, $a(b+c) = 0 = 0 + 0 = ab + ac$.\\
\\
\prob{1.1.18.}
\begin{enumerate}
	\item $R$ is an $R$-module. For $r, s, x, y\in R$,
	\begin{enumerate}
		\item $r(x+y) = rx+ry$ ($R$ is a ring)
		\item $(r+s)x = rx+sx$ ($R$ is a ring)
		\item $r(sx) = (rs)x$ (associativity)
	\end{enumerate}
	\item No. For $r, s, x, y \in R$,
	\begin{enumerate}
		\item \textit{Is $(R, +, \gop)$ a ring?} Yes.
		\item \textit{Is $(R, +, \sang)$ an $R$-module?} Yes.
		\item But $(rx)y$ may not equal $x(ry)$, since commutativity might not hold.
	\end{enumerate}
\end{enumerate}~
\\
\prob{1.1.19.} Show that $\mf{L}(V, V)$ is an $\R$-algebra.
\begin{enumerate}
	\item $(\mf{L}(V, V), +, \gop)$ is a ring. For $L, M, N \in \mf{L}(V, V)$,
	\begin{enumerate}
		\item $(L+M)N = LN + MN$, $L(M + N) = LM + LN$ (evaluate at $v\in V$ to check)
		\item $(\mf{L}(V, V), +)$ is an abelian group under addition. (composition is associative)
	\end{enumerate}
	\item $(\mf{L}(V, V), +, \sang)$ is an $\R$-module. For $r, s\in \R$, $L, M\in \mf{L}(V, V)$,
	\begin{enumerate}
		\item For $v\in V$, $r(L + M)(v) = rL(v) + rM(v) = (rL + rM)(v)$, thus $r(L + M) = rL + rM$.
		\item For $v\in V$, $(r+s)L(v) = rL(v) + sL(v) = (rL + sL)(v)$, thus $(r+s)L = rL + sL$.
		\item For $v\in V$, $r(sL)(v) = rsL(v) = (rs)L(v)$, thus $r(sL) = (rs)L$.
	\end{enumerate}
	\item For $r\in \R$, $L, M \in \mf{L}(V, V)$, $(rL)M = r(LM) = L(rM)$ (evaluate at $v\in V$ to check) 
\end{enumerate}
우리는 $\mf{L}(V, V)$ 를 \textbf{endomorphism algebra} on $V$ 라고 부른다.\\
\\
\prob{1.1.20.}
\begin{enumerate}
	\item $\R^n$ is an $\mf{M}_{n, n}(\R)$-module
	\begin{enumerate}
		\item $\mf{M}_{n, n}(\R)$ is a ring, and $(\R^n, +)$ is an abelian group under addition.
		\item $A(X+Y) = AX+AY$ \quad ($A\in \mf{M}_{n, n}(\R)$, $X, Y\in \R^n$)
		\item $(A+B)X = AX+BX$ \quad ($A, B\in \mf{M}_{n, n}(\R)$, $X\in \R^n$)
		\item $A(BX) = (AB)X$ \quad ($A, B\in \mf{M}_{n, n}(\R), X\in \R^n$)
	\end{enumerate}
	\item $V$ is an $\mf{L}(V, V)$-module
	\begin{enumerate}
		\item $\mf{L}(V, V)$ is a ring, and $V$ is an abelian group under addition. (vector space)
		\item $L(v+w) = L(v)+L(w)$ \quad ($L\in \mf{L}(V, V)$, $v, w\in V$)
		\item $(L + M)(v) = L(v) + M(v)$ \quad ($L, M\in \mf{L}(V, V)$, $v\in V$)
		\item $L(M(v)) = (LM)(v)$ \quad ($L, M \in \mf{L}(V, V)$, $v\in V$)
	\end{enumerate}
\end{enumerate}
We say that $V$ is an $\mf{L}(V, V)$-module with respect to the \textbf{natural action} of $\mf{L}(V, V)$. In short, $\mf{L}(V, V)$ \textbf{acts naturally} on $V$.\\
\\
\prob{1.1.21.} First, we know that $F[t]$ is a ring under multiplication, and $V$ is an abelian group under addition. For $v, w\in V$, $f(t), g(t)\in F[t]$,
\begin{enumerate}
	\item $f(t)\cdot (v+w) = f(T)(v+w) = f(T)(v)+f(T)(w) = f(t)\cdot v + f(t)\cdot w$
	\item $(f(t)+g(t))\cdot v = (f(T) + g(T))(v) = f(T)(v) + g(T)(v) = f(t)\cdot v + g(t)\cdot v$
	\item $(f(t)g(t))\cdot v = (f(T)g(T))(v) = f(T)(g(T)(v)) = f(T)(g(t)\cdot v) = f(t)\cdot (g(t)\cdot v)$
\end{enumerate}
Thus $V$ is an $F[t]$-module.\\
\\
\prob{1.1.22.} No, $\R$ is not an $\R[t]$-module. For $\alpha, \beta\in \R$ and $f(t)\in \R[t]$, $f(t)\cdot(\alpha + \beta) = f(\alpha + \beta)$, and $f(t)\cdot \alpha + f(t)\cdot \beta = f(\alpha) + f(\beta)$.
But generally, $f(\alpha + \beta) \neq f(\alpha) + f(\beta)$. The difference between the previous exercise is that $f(T) : V\ra V$ is a \textit{linear} operator, whereas (function) $f: \R \ra \R$ is non-linear in general.

\subsection{Elementary Properties of Rings}
\prob{1.2.2.} Use associativity and commutativity.
\begin{enumerate}
	\item $-a-b + a + b = -a +(-b) + a + b = -a + a + b + (-b) = (-a+a) + (b+(-b)) = 0 + 0 = 0$. Therefore $-a-b$ is the inverse of $a+b$. Thus $-(a+b) = -a-b$.
	\item $a + (-a) = 0$. The inverse of $-a$ is $a$. Thus $-(-a) = a$.
	\item (Induction) For $n = 0, 1$, trivial. For $n\geq 1$, suppose $-(na) = (-n)a$, which means that $(-n)a + na = 0$. Then $(n+1)a + (-(n+1))a = na + a + (-n)a + (-a)$, by associativity. Since $A$ is abelian, this is equal to $na + (-n)a + a + (-a) = 0 + 0 = 0$. Thus $-(n+1)a = (-(n+1))a$. For $n < 0$, substitute $m = -n > 0$. Then we need to show that $-(-ma) = ma$. The equation follows directly from (2), and the given property holds for all $n\in \Z$.
	\item No, Consider an abelian group $(\{0, x\}, +)$ ($x\neq 0$), with operations defined as
	$$0 + 0 = 0,\; 0 + x = x,\; x + 0 = x,\; x + x = 0$$
	Then $x + x = 2x = 0$, but $x\neq 0$. Also consider $\mu_3 = \{x \mid x^3 = 1, x \in \C\}$, then $3x = 0$, but $x $ need not be $0$.
\end{enumerate}