\subsection{`Additional Relation'}
\prob{4.5.20.} First, deduce that \(\Z[\i]/\) will be isomorphic to \(\Z_{10}\) because \(1+3\i = 0\) gives `\(10 = 0\)'. Now define a ring homomorphism
\[
	\psi = \pi \times \imath_\Z : \Z \ra \Z[\i] \ra \Z[\i]/\ideal{1+3\i}
\]
We will show that \(\ker \psi = 10\Z\). It is enough to show that \(\ker \psi \leq 10\Z\). For any \(a\in \ker\psi\), \(a\) is also an element of \(\ideal{1+3\i}\), thus there exists some \(m+n\i\in \Z[\i]\) such that \(a = (1+3\i)(m+n\i)\). Then we have
\[
	a = (m-3n) + (3m + n)\i\in \Z \implies a = m - 3n, 3m+n = 0
\]
Thus \(a = m - 3n = m - 3(-3m) = 10m \in 10\Z\), therefore \(\ker\psi  = 10\Z\). Next we show that \(\psi\) is surjective. From the relation \(1+3\i = 0\), multiplying \(\i\) on both sides give \(\i = 3\). Then any element \(m+n\i \in \Z[\i]\) can be written as \(m+3n\in \Z\). Thus \(\psi\) is surjective, and by the First Isomorphism Theorem,
\[
	\Z_{10} = \Z/\ker\psi \approx \im\psi = \Z[\i]/\ideal{1+3\i}
\]
(\textbf{Generalization}) If \(\gcd(a, b) = 1\), the following holds.
\[
	\Z[\i]/\ideal{a+b\i} \approx \Z_{a^2+b^2}
\]