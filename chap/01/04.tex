\subsection{Elementary Properties of Modules}
When we mention an $R$-module, we implicitly think that $1 \in R$. Moreover, $R$-module $M$ satisfies the condition

\begin{enumerate}
	\item[{\sffamily (M5)}] If $x\in M$, then $1x = x$.
\end{enumerate}~
\\
\defn{1.4.2.} If $F$ is a \textbf{division ring}, we call an $F$-module an $F$-\textbf{vector space}.\footnote{Note that we don't need commutativity in the definition of a module. So $F$ doesn't have to be a field.}\\
\\
\prob{1.4.3.}
\begin{enumerate}
	\item $0x = (0+0)x = 0x + 0x$. Thus $0x = 0$.
	\item $r0 = r(0+0) = r0 + r0$. Thus $r0 = 0\in M$.
	\item $(-1_R)x + x = (-1_R)x + 1_R\cdot x = (-1_R+1_R)x = 0x = 0$. Thus $-x = (-1_R)x$.
	\item $(-r)x + rx = (-r+r)x = 0x = 0$. Thus $-(rx) = (-r)x$.
	\item $rx + r(-x) = r(x + (-x)) = r0 = 0$. Thus $r(-x) = -(rx)$.
\end{enumerate}~
\\
\defn{1.4.5.} Let $M$ be an $R$-module.
\begin{enumerate}
	\item For $x\in M$, if there exists a \textbf{non-zero} element $r\in R$ such that $rx = 0$, $x$ is called an $r$-\textbf{torsion element}. And we say that ``$r$ \textbf{kills} $x$" or ``$r$ \textbf{annihilates} $x$".
	\item We define the \textbf{torsion part} of $M$ as $$M\tor = \{x\in M \mid r_x\cdot x = 0 \text{ for some non-zero } r_x\in R\}$$
	\item If $M\tor = M$, $M$ is a \textbf{torsion $R$}-\textbf{module}. If $M\tor = \{0\}$, $M$ is a \textbf{torsion-free $R$}-\textbf{module}.
	\item If $N\subseteq M$, we define the \textbf{annihilator ideal} of $N$ as
	$$\ann(N) = \ann_R(N) = \{r\in R\mid rx = 0 \text{ for all } x\in N\}$$
	and if $S\subseteq R$, we define $\Ann(S)$ as
	$$\Ann(S) = \Ann_M(S) = \{x\in M \mid rx = 0 \text{ for all }r\in S\}$$
\end{enumerate}~
\\
\prob{1.4.6.}
\begin{enumerate}
	\item Since $r\in R\cross$, there exists an inverse. If $rx = 0$, $x = r\inv \cdot 0 = 0$. Thus $\Ann_M(r) = \{0\}$.
	\item Because $F$ is a division ring. So if $r_x\cdot x = 0$ for some non-zero $r_x\in R$, $\exists r_x\inv$, and $x = 0$. Thus $M\tor = \{0\}$ and $F$-vector space is torsion-free.
	\item $\Ann_{\R^n}(A) = \{x \in \R^n \mid Ax = 0\} = \ker L_A$.
\end{enumerate}~
\\
\prob{1.4.7.}
\begin{enumerate}
	\item $\ann_{F[t]}(V) = \{f(t)\in F[t]\mid f(T)\cdot v = 0 \text{ for all } v \in V\}$. Since $f(T)\cdot v = 0$ for all $v\in V$ if and only if $f(T) = 0$, $\ann_{F[t]}(V) = \mc{I}_T$.
	\item Trivial.
	\item $V\tor = \{v\in V\mid f_v(T)\cdot v = 0 \text{ for some non-zero } f_v(t)\in F[t]\}$. Take $f_v(t) = \phi_T(t)$ then $\phi_T(T)\cdot v = 0$ for all $v\in V$, by Cayley-Hamilton Theorem. Thus $V\tor = V$, and $V$ is a torsion $F[t]$-module.
\end{enumerate}~
\\
\obs{1.4.8.} \textbf{Abelian group} = \textbf{$\Z$-module}.\\
\\
\prob{1.4.9.}
\begin{enumerate}
	\item Suppose a finite abelian group $A$ has $k$ elements. It is trivial that $0 \in A\tor$. So suppose for any non-zero $a\in A$, consider the elements $1a, \dots, ka$. These are elements of $A$ and there are $k$ of them. We have two cases.
	\begin{enumerate}
		\item If $na = 0$ for some $n = 1, 2, \dots, k$, then $a \in A\tor$.
		\item If $na \neq 0$ for all $n$, by the pigeonhole principle, there exists $i, j\in \{1, 2, \dots, k\}$ such that $ia = ja$. ($i\neq j$) Then $(i - j)a = 0$, thus $a\in A\tor$.
	\end{enumerate}
	Therefore $A\subseteq A\tor$, (and since $A\tor\subseteq A$) $A = A\tor$, and finite abelian group is a torsion $\Z$-module.
	\item $\Z$ is a torsion-free $\Z$-module because $\Z$ is an integral domain. For $n\in \Z$, if there were some non-zero $m_n \in \Z$ such that $m_n \cdot n = 0$, $n$ must be 0. Thus $\Z\tor = \{0\}$.
	\item For $(k, \overline{m})\in (\Z\times \Z_n)\tor$, suppose $0\neq \alpha \in \Z$ exists, such that $\alpha \cdot (k, \overline{m}) = 0$. Then $\alpha k =0$ and $\alpha \overline{m} = 0$. Since $\Z$ is an integral domain, we directly see that $k = 0$, and choosing $\alpha = n$ gives $\alpha\overline{m} = n\overline{m} = \overline{0}$.\footnote{We need $n>1$ here.} Now we have $(k, \overline{m})\in \{0\}\times \Z_n$. Therefore, $(\Z\times \Z_n)\tor \subseteq \{0\}\times \Z_n$.\\
	We also see that $\{0\}\times \Z_n \subseteq (\Z\times\Z_n)\tor$ because $(0, \overline{m}) \in (\Z\times\Z_n)\tor$.
\end{enumerate}~
\\
\prob{1.4.10.}
\begin{enumerate}
	\item For two $R$-modules $M_1$, $M_2$, let $x_1, x_2\in M_1$, $y_1, y_2\in M_2$, and $r\in R$. On $M_1\times M_2$, define addition as
	$$(x_1, y_1) + (x_2, y_2) = (x_1+x_2, y_1+y_2)$$
	and $R$-multiplication as
	$$r(x_1, y_1) = (rx_1, ry_1)$$
	Then it is easy to check that $M_1\times M_2$ is indeed an $R$-module.
	\item For two $R$-algebras $\mc{A}$, $\mc{B}$, let $a_1, a_2\in \mc{A}$, $b_1, b_2\in \mc{B}$, and $r\in R$. On $\mc{A}\times \mc{B}$, define addition as
	$$(a_1, b_1) + (a_2, b_2) = (a_1+a_2, b_1+b_2)$$
	, multiplication as
	$$(a_1, b_1)\cdot(a_2, b_2) = (a_1a_2, b_1b_2)$$
	and finally $R$-multiplication as
	$$r(a_1, b_1) = (ra_1, rb_1)$$
	Then it is easy to check that $\mc{A}\times\mc{B}$ is indeed an $R$-algebra.
\end{enumerate}~
\\
\prob{1.4.12.} Define $\Z$-multiplication on ring $R$ as notation {\sffamily 1.2.1}. Then $(R, +, \cdot)$ is a $\Z$-module, and since $R$ is already a ring, $R$ is a $\Z$-algebra. Conversely, if you ignore the $\Z$-multiplication on a $\Z$-algebra, $\Z$-algebra is already a ring.\\
\\
\prob{1.4.13.} We have previously shown that $R$ itself is a ring, and an $R$-module. Since $R$ is a commutative ring with 1, for $a, b, c\in R$, $a(bc) = (ab)c = b(ac)$ holds. Thus $R$ is an $R$-algebra.\\
\\
\prob{1.4.16.}
\begin{enumerate}
	\item Let $f(t), g(t)\in R[t]$. Suppose the leading coefficients of $f(t)$, $g(t)$ are $a_n$ and $b_m$, respectively. ($a_n, b_m\neq 0$) If $f(t)g(t) = 0$, $a_nb_m$ should be 0, and since $R$ is an integral domain, $a_nb_m \neq 0$. Thus $R[t]$ cannot have any zero divisors.
	\item No. In $\Z/4\Z = \Z_4$, consider $f(t)=1+2t$. Then $f(t)^2 = 1$, so $f(t)\in \Z_4[t]\cross$, but $f(t)\notin \Z_4\cross$.
	\item Let $f(t) \in R[t]\cross$. Suppose $\deg f(t) \geq 1$. Then $f(t) = \sum_{i=0}^n a_i t^i$ with $a_i\in R$ and $a_n \neq 0$. Let the inverse of $f(t)$ be $g(t)\in R[t]$.
	\begin{enumerate}
		\item If $\deg g(t) = 0$, $g(t)$ is a constant, and $f(t)g(t)$ cannot equal 1. (Contradiction)
		\item If $\deg g(t) \geq 1$, let $g(t) = \sum_{j=0}^m b_j t^j$ with $b_j\in R$ and $b_m \neq 0$. Write 
		$$f(t) = a_0 + tf_1(t) \quad g(t) = b_0 + tg_1(t)$$
		, where $f_1(t), g_1(t)\in R[t]$. Then $$f(t)g(t) = a_0b_0 + b_0tf_1(t) + a_0tg_1(t) + t^2f_1(t)g_1(t)$$
		since $f(t)g(t) = 1$, $a_0b_0 = 1$, $b_0f_1(t) = a_0g_1(t) = f_1(t)g_1(t) = 0$. But since $R$ is an integral domain, we have $a_nb_m\neq 0$ which gives $f_1(t)g_1(t) \neq 0$. (Contradiction)
	\end{enumerate}
	Thus $\deg f(t) = 0$, and $f(t) = a_0 \in R$. Now the inverse of $f(t)$ exists if and only if $a_0$ has an inverse. We have proven that $R[t]\cross \subseteq R\cross$. Since $R\cross \subseteq R[t]\cross$ is trivial, we arrive at the conclusion that $R[t]\cross = R\cross$.
\end{enumerate}~
\\
\prob{1.4.17.} Just multiply like normal polynomial multiplication.\\
\\
\prob{1.4.18.} Let $A = \begin{pmatrix}
a & b \\ c & d
\end{pmatrix} \in \mf{M}_{2, 2}(\Z)$. If $\det A = \pm 1$, $A\inv = \begin{pmatrix}
d & -b \\ -c & a
\end{pmatrix} \in \mf{M}_{2, 2}(\Z)$. Thus $A\in \mf{M}_{2, 2}(\Z)\cross$. Conversely, if $A\in \mf{M}_{2, 2}(\Z)\cross$, an inverse of $A$ exists, and $A\cdot A\inv = I_2$. From the properties of determinants, $1 = \det (A\cdot A\inv) = (\det A)(\det A\inv)$ and since $A, A\inv \in \mf{M}_{2, 2}(\Z)$, the determinants should be an integer. Thus $\det A = \pm 1$.
$$\mf{M}_{2, 2}(\Z)\cross = \{A\in \mf{M}_{2, 2}(\Z) \mid \det A = \pm 1\}$$

\pagebreak