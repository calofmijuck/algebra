\subsection{Universal Property I}
\defn{1.7.1.} 다음 구문
\begin{center}
	\textit{For every ......, there exists a unique homomorphism such that ......}
\end{center}
으로 묘사되는 성질을 \textbf{universal property} 라고 부른다.\\
\\
\prop{1.7.2.} For every ring $R$ with 1, there exists a unique ring homomorphism $\mu:\Z\ra R$ such that $\mu(1)=1$.\\
\pf. From the definition of ring homomorphism, it must be the case that $\mu(n) = n\cdot 1_R$ for $n\in \Z$!\\
\\
\prob{1.7.3.} For every group $G$ and for every element $g\in G$, there exists a unique group homomorphism $\gamma_g:\Z\ra G$ such that $\gamma_g(1) = g$.\\
\pf. Similarly, it must be the case that $\gamma_g(n) = g^n$ for $n\in \Z$!\\
\\
\prob{1.7.4.}
\begin{itemize}
	\item For (1)$\sim$(6), the answer is no.
	\item[(7)] Define $\varphi:\Z_4\ra \Z_2$ as $\varphi(x) = 0$ if $x = 0, 2$ and $\varphi(x) = 1$ otherwise. (Parity)
	\item[(8)] No. Look at the dimensions of the vector space.
\end{itemize}~
\\
\prob{1.7.5.}
\begin{enumerate}
	\item Just check! Do you remember \textit{magic bars}?
	\item Similar to (1).
	\item Since $\Q$ is a division ring, check that $\Q(\sqrt{2})$ is a $\Q$-module. It will also be a $\Q$-algebra. The $\Q$-basis will be $\{1, \sqrt{2}\}$ and $\dim_\Q \Q(\sqrt{2}) = 2$.
\end{enumerate}~
\\
\prob{1.7.6.}
\begin{enumerate}
	\item Suppose there exists a ring homomorphism $\psi$, other than $0$. Recall that $\Z[\rmbf{i}]$ is an integral domain. Thus $\psi(1) = 1$.\footnote{Problem 1.6.11 (1)} By induction we can prove that $\psi(na) = n\psi(a)$ for $n\in \Z$, $a\in \Z[\rmbf{i}]$. Now consider $\rmbf{i}$. Since $-1=\psi(-1) = \psi(\rmbf{i}\cdot \rmbf{i}) = \psi(\rmbf{i})\psi(\rmbf{i})$, let $\psi(\rmbf{i}) = x + y\rmbf{i}$ where $x, y\in \Z$ and solve $(x+y\rmbf{i})^2 = -1$. Then we get $xy = 0$. If $y = 0$, $x\notin\Z$ so $x = 0$ and $y = \pm 1$. Thus $\psi(\rmbf{i}) = \pm \rmbf{i}$.\\
	Finally, since $\psi$ is a ring homomorphism, $\varphi(m + n\rmbf{i}) = \varphi(m) + \varphi(n\rmbf{i}) = m + n\varphi(\rmbf{i}) = m \pm n\rmbf{i}$ for any $m, n\in \Z$. Thus $\psi$ is either $id$ or $\varphi$.
	\item Similar to (1). 
\end{enumerate}~
\\
We notice that we only need to check the outputs of $\varphi$ (homomorphism) for the elements in the given basis. The outputs will determine $\varphi$ uniquely. This is somewhat similar to the \textbf{linear extension theorem}...\\
\\
\thm{1.7.9.} (\textbf{Linear Extension Theorem}) Let $V$ be an $F$-vector space with an $F$-basis $\mf{B} = \{v_i \mid i\in I\}$. Then for every $F$-vector space $W$ and for every subset $\{w_i\mid i\in I\}$ of $W$, there exists a unique $F$-linear map $\varphi:V\ra W$ such that $\varphi(v_i) = w_i$ for all $i\in I$.\\
\\
이 삼각형은 함수 $f:\mf{B}\ra W$ 를 linear map $\varphi: V\ra W$ 로 \textbf{확장}했다는 의미이다. $\imath$ 는 inclusion map. 오른쪽 삼각형은 다음 명제를 해석한 것이다.
$$
\begin{tikzcd}
\mf{B} \arrow{rr}{\imath} \arrow[swap]{dr}{f} & \arrow[d, phantom, "\circlearrowright"]& V\arrow{dl}{\varphi} \\
& W  &
\end{tikzcd} \qquad
\begin{tikzcd}
S \arrow{rr}{\imath} \arrow[swap]{dr}{f} & \arrow[d, phantom, "\circlearrowright"]& \Z\arrow{dl}{\varphi} \\
& G  &
\end{tikzcd}
$$
\\
\prop{1.7.10.} Put $\{1\}\subset\Z$. Then for every group $G$ and for every function $f:S\ra G$, there exists a unique group homomorphism $\varphi: \Z\ra G$ such that $\varphi\circ \imath = f$.\\
\\
\prob{1.7.11.} For every $R$-algebra $\mc{A}$ with $1_\mc{A}$, and for every $\alpha\in \mc{A}$, there exists a unique $R$-algebra homomorphism $\mc{E}_\alpha:R[t]\ra\mc{A}$ such that $\mc{E}_\alpha(t) = \alpha$ and $\mc{E}_\alpha(1)=1$.\\
This homomorphism $\mc{E}_\alpha$ is in fact the \textbf{evaluation homomorphism} defined as $\mc{E}_\alpha(f(t)) = f(\alpha)$.
\begin{itemize}
	\item Check that $\mc{E}_\alpha(n) = n\cdot 1_\mc{A}$ for $n\in \Z$.
	\item Check that $\mc{E}_\alpha(nt) = n\alpha$ for $n\in \Z$.
	\item Check that $\mc{E}_\alpha(t^n) = \alpha^n$ for $n\in \N$.
\end{itemize}~
\\
\defn{1.7.13.}
\begin{enumerate}
	\item $\mc{A}$ 는 $R$-algebra with 1 이고, $\alpha\in \mc{A}$ 일 때, $\mc{E}_\alpha:R[t]\ra \mc{A}$ 를
	$$\mc{E}_\alpha\big(f(t)\big) = f(\alpha) \qquad (f(t)\in R[t])$$
	로 정의하면, $\mc{E}_\alpha$ 는 $R$-algebra homomorphism 이다. $\mc{E}_\alpha$ 를 \textbf{evaluation homomorphism} at $\alpha$ 라고 부른다.
	\item $f(t)\in R[t]$ 일 때, $a\in R$ 이 $$\mc{E}_\alpha\big(f(t)\big) = f(\alpha) = 0$$
	을 만족하면 $\alpha$ 를 $f(t)$ 의 \textbf{root} 라고 부른다. 이 때 $R$ 자신을 $R$-algebra 로 본다.
\end{enumerate}~
\pagebreak\\
\prob{1.7.14.} (\textbf{Division Algorithm in $R[t]$}) \textit{Let $R$ be a commutative ring with 1 and $f(t), g(t)\in R[t]$. If the leading coefficient of $g(t)$ is a unit, there exists unique $q(t), r(t)\in R[t]$ such that
	$$f(t) = q(t)g(t)+r(t) \qquad (\deg r < \deg g)$$
}
\pf. (\textit{Existence})
\begin{itemize}
	\item If $f(t) = 0$, set $q(t) =r(t) = 0$.
	\item If $\deg f < \deg g$, then $q(t) = 0$ and $r(t) = f(t)$.
	\item Now let $\deg f \geq \deg g$. We will use induction on the degree of $f$.
	\begin{enumerate}
		\item[(i)] If $\deg f = 0$, then $\deg g = 0$.\footnote{You can't divide by 0, so $\deg g \neq -\infty$.} Let $f(t) = a$, $g(t) = b$. ($a, b\neq 0$ and $a, b\in R$) Since $b\in R\cross$, write $a = (ab\inv)b + 0$. Then $q(t) = ab\inv$ and $r(t) = 0$.
		\item[(ii)] Suppose the proposition holds for $\deg f < n$.
		\item[(iii)] If $\deg f = n$, let
		$$f(t) = a_nt^n + \cdots + a_1t + a_0,\; g(t) = b_mt^m + \cdots + b_1t + b_0 \in R[t]$$
		where $a_n, b_m\neq 0$ and $m\leq n$. Since $b_m\in R\cross$,
		$$a_nb_m\inv t^{n-m}g(t) = a_nt^n + a_nb_m\inv b_{m-1}t^{n-1} + \cdots + a_nb_m\inv b_1 t^{n-m+1}+a_nb_m\inv b_0 t^{n-m}$$
		Now consider $h(t) = f(t) - a_nb_m\inv t^{n-m} g(t)$. Note that $\deg h < n = \deg f$. By induction hypothesis, there exists $q_1(t), r(t)\in R[t]$ such that
		$$h(t) = q_1(t)g(t) + r(t) \qquad (\deg r < \deg g)$$
		Therefore, we have
		$$f(t) = (a_nb_m\inv t^{n-m} + q_1(t))g(t) + r(t) \qquad (\deg r < \deg g)$$
		Thus we have shown the existence of $q(t)$ and $r(t)$.
	\end{enumerate}
\end{itemize}~\\
(\textit{Uniqueness}) Suppose that there exists $q(t), q'(t), r(t), r'(t)\in R[t]$ such that 
$$f(t) = q(t)g(t)+r(t) = q'(t)g(t)+r'(t)$$
and $q(t)\neq q'(t)$, $r(t)\neq r'(t)$.
Then $g(t)(q(t) - q'(t)) = r'(t)-r(t)$. Now we compare the degrees on each side. Since $\deg r, \deg r' < \deg g$ we have $\deg (r'-r)<\deg g$. Moreover $\deg (q-q') \geq 0$, so
$$\deg(g(q-q')) = \deg g + \deg(q - q') \geq \deg g > \deg (r' - r)$$
which is a contradiction. Thus $q(t) = q'(t)$ and $r(t)=r'(t)$ directly follows. Thus we have shown that $q(t), r(t)$ are unique.
