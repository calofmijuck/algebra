\subsection{Fields and Integral Domains}
\defn{1.3.1.} $F$ 가 \textbf{commutative} [ring with 1] 일 때, $F$ 가 다음 조건
\begin{enumerate}
	\item[{\sffamily (F1)}] $F\cross = F - \{0\}$
\end{enumerate}
을 만족하면, $F$ 를 \textbf{field}(\textbf{체}) 라고 부른다. 이는 당연히
\begin{enumerate}
	\item[\sffamily (F1)'] $F$ 의 모든 \textbf{non-zero} element 는 invertible
\end{enumerate}
과 동치이다.\\
\\
\prob{1.3.2.} $F - \{0\}$ is commutative with respect to multiplication, and since $0$ always satisfies commutativity, $F$ is a commutative ring with 1. Now, from {\sffamily (f2)}, we know that every element in $F - \{0\}$ is invertible. (Every element in a group has an inverse) By {\sffamily (F1)'}, $F$ is a field.\\
\\
\prob{1.3.3.} \textbf{Typical examples of field}
\begin{enumerate} 
	\item $1\in \Q$, commutative, $\Q\cross = \Q - \{0\}$
	\item $1\in \R$, commutative, $\R\cross = \R - \{0\}$
	\item $1\in \C$, commutative, $\C\cross = \C - \{0\}$
	\item $1\in \Q(\sqrt{2})$, commutative, $\Q(\sqrt{2})\cross = \Q - \{0\}$
	\item $1\in \F_2$, commutative by definition, 0 is not invertible. $\F_2\cross = \F_2 - \{0\}$
\end{enumerate}~
\\
$\F_2$ is the \textbf{finite field} of order 2.\\
\\
\prob{1.3.4.} By the definition of additive/multiplicative identity, we can fill out the following Cayley table.
\begin{center}
	\begin{tabular}{c|cc}
		$+$ & 0 & 1 \\ \hline
		0 & 0 & 1\\
		1 & 1 & \\
	\end{tabular} \qquad
	\begin{tabular}{c|cc}
		$\times$ & 0 & 1 \\ \hline
		0 & & 0\\
		1 & 0 & 1\\
	\end{tabular}
\end{center}
We only need to define the result for $1+1$ and $0 \times 0$. If $1+1=1$, then the element 1 would not have an additive inverse, so $1+1$ should be 0. Also if $0\times 0 = 1$, then $0 = 0 \times 1 = 0\times(0+1) = 0\times 0 + 0\times 1 = 1 + 0 = 1$, contradicting $0\neq 1$. Thus $0\times 0 = 0$, and we can check that all the other properties hold.
\begin{center}
	\begin{tabular}{c|cc}
		$+$ & 0 & 1 \\ \hline
		0 & 0 & 1\\
		1 & 1 & 0\\
	\end{tabular} \qquad
	\begin{tabular}{c|cc}
		$\times$ & 0 & 1 \\ \hline
		0 & 0 & 0\\
		1 & 0 & 1\\
	\end{tabular}
\end{center}~
\\
\prob{1.3.5.} ...\\
\\
\defn{1.3.6.} Field $F$ 의 정의에서 $F$ 의 commutativity 조건을 제외하면, $F$ 를 \textbf{division ring} 이라고 부른다. 그리고, non-commutative division ring 은 \textbf{skew-field} 라고 부른다.\\
\\
\defn{1.3.8.} $a, b\neq 0$ 이 ring $R$ 의 원소일 때, 만약 $ab=0$ 이면, $a$ 와 $b$ 를 \textbf{zero divisor} 라고 부른다.\\
\\
\prob{1.3.9.}
\begin{enumerate}
	\item For any non-zero nilpotent element $a \in R$, suppose $m$ is the smallest natural number such that $a^m = 0$. Then $a, a^{m-1}\neq 0$ but $a^m = 0$, and $a$ is indeed a zero divisor.
	\item Suppose $a$ is a zero divisor. Then there exists two non-zero elements $a, b$ such that $ab = 0$. Since $a\in R\cross$, there exists $a\inv$. Multiplying $a\inv$ on the left gives $a\inv\cdot ab = (a\inv a)b = 1\cdot b = b = 0$, contradicting that $b$ is non-zero. Thus $a$ is not a zero divisor.
\end{enumerate}~
\\
\prob{1.3.10.}
\begin{enumerate}
	\item (\mimp) 1.3.9 (2)\\
	(\mimpb) Suppose $0\neq A\notin \mf{M}_{n, n}(\R)$. Then $A\inv$ does not exist. Then the columns of $A$ are linearly dependent, meaning that there exists $x\neq 0$ such that $Ax=0$. Consider another matrix $B$ where all its columns are $x$. Then $B\neq 0$, but $AB = 0$, contradicting that $A$ is not a zero divisor.
	\item When $R$ is finite, the answer is no. Suppose a non-zero element $a\in R$ exists, which is neither a zero divisor nor a unit. Consider a map $x\mapsto ax$ for all $x\in R$. If this map is injective, it has to be surjective. Then There exists $x\in R$ such that $ax = 1$, contradicting that $a$ is not a unit. If this map is not injective, there exists $u, v\in R$ ($u\neq v$) such that $au = av$. Then $a(u-v) = 0$, contradicting that $a$ is not a zero divisor. Thus a non-zero element is either a unit or a zero divisor.\\
	When $R$ is infinite, the answer is yes. Consider $2\in \Z$. $2$ is not a unit, and it is also not a zero divisor. 
\end{enumerate}~
\\
\prob{1.3.11.}
\begin{enumerate}
	\item Commutativity and finiteness is trivial from the operation table. And indeed, $a(b+c)=ab+ac$, $(a+b)c = ac+bc$ holds for all $a, b, c\in \Z_4$. Also, $\overline{1}$ is the multiplicative identity.
	\item $\overline{2}\cdot \overline{2} = \overline{0}$ but $\overline{2} \neq \overline{0}$. Thus $\overline{2}$ is a zero divisor.
	\item $4 = 1_R + 1_R + 1_R + 1_R = \overline{1} + \overline{1} + \overline{1} +\overline{1} = \overline{2} + \overline{1} + \overline{1} = \overline{3} + \overline{1} = \overline{0} = 0$.
	\item Multiplication is defined by $\overline{a}\cdot \overline{b}$ = (remainder of $ab$ divided by 4).
\end{enumerate}~
\\
\defn{1.3.13.} $1\in R$ 일 때, $0=n=n\cdot 1_R \in R$ 인 최소의 자연수 $n > 1$ 을 ring $R$ 의 \textbf{characteristic} 이라 부르고, $\ch(R) = n$ 으로 표기한다.\footnote{$n>1$ because if $n = 1$, $0=1$.} 만약 이러한 자연수 $n$ 이 존재하지 않으면, $R$ 은 \textbf{characteristic zero} 라고 하고, $\ch(R) = 0$ 으로 표기한다.\\
\\
\defn{1.3.14.} $1\in R$ 일 때, additive group $(R, +)$ 에서 $1_R$ 의 order $\abs{1_R}$ 이 finite 이면 $\ch(R) = \abs{1_R}$ 으로 정의한다. 한편, $\abs{1_R}=\infty$ 이면, $\ch(R)=0$ 으로 정의한다.\\
\\
\prob{1.3.15.}\\
(\mimp) For all $a\in R$, $na = (n\cdot 1_R)a =0\cdot a = 0$.\\
(\mimpb) Since $na = 0$ for all $a\in R$, $n\cdot 1_R = 0$.\\
\\
\prob{1.3.16.}\\
((1)\miff(2)) Holds by definition.\\
((2)\miff(3)) Since $a+a = 2a$, the result holds by 1.3.15.\\
((2)\miff(4)) Move the term $1_R$ to show the result.\\
\\
\prob{1.3.17.}
\begin{enumerate}
	\item For $\Z, \Q, \R, \C$, $n \cdot 1_R = n$ will never be 0, for all $n > 1$. Thus $\ch(\Z) =\ch(\Q) =\ch(\R) = \ch(\C) = 0$.
	\item Suppose $\ch(F) = n$, then $n\cdot 1_F = 0$. Since $1_{F[t]} = 1\in F$, we have $n\cdot 1_{F[t]} = 0$. Also because $1_{\mf{M}_{n, n}(F)} = \diag(1_F, \dots, 1_F) = I_n$, we have $n\cdot I_n = \diag(n\cdot 1_F, \dots, n\cdot 1_F) = 0$. Thus $\ch(F) = \ch(F[t]) = \ch(\mf{M}_{n, n}(F)) = n$. (For the proofs of $\ch(F[t]) = \ch(\mf{M}_{n, n}(F)) = n$, the minimality condition holds. If there exists $n > m\in \N - \{1\}$ s.t. $m\cdot 1_F = 0$, it contradicts $\ch(F) = n$.)
	\item $\overline{1} + \overline{1} = 2\cdot 1_{\bb{F}_2} = 0$, thus $\ch(\F_2) = 2$.
	\item From the operation table, it is obvious that $4\cdot \overline{1} = 0$, thus $\ch(\Z_4) = 4$.
\end{enumerate}~
\\
\defn{1.3.18.} \textbf{Commutative} ring with 1 $D$ 가 zero divisor 를 갖지 않으면, $D$ 를 \textbf{integral domain} 이라고 부른다.\\
\\
\prob{1.3.19.} \textbf{Typical examples of integral domain}
\begin{enumerate}
	\item $\Z$ is a commutative ring with 1. For $a, b\in \Z$, suppose $ab = 0$. If $a\neq 0$ and $b\neq 0$, $ab \neq 0$, contradicting that $ab=0$. Thus $a=0$ or $b=0$.
	\item $\Z[\bf{i}]$ is a commutative ring with 1. For $a, b\in \Z[\bf{i}]$, suppose $ab = 0$. Let $a = x_1+y_1\bf{i}$, $b = x_2+y_2\bf{i}$. ($x_i, y_i\in \Z$) Then $0 = ab = (x_1x_2-y_1y_2) + (x_1y_2+x_2y_1)\bf{i}$. Thus $x_1x_2 = y_1y_2$, $x_1y_2 + x_2y_1= 0$. Checking all possible cases gives $a = 0$ or $b=0$.
	\item A field is a commutative ring with 1. For $a, b\in F$, suppose $ab = 0$. If $a\neq 0$ and $b\neq 0$, there exists an inverse of $a$. Thus $a\inv \cdot a \cdot b = a\inv \cdot 0 = 0$, which gives $b=0$. This contradicts the assumption $b\neq 0$. Therefore $a = 0$ or $b = 0$.
	\item $D[t]$ is a commutative ring with 1. For $f(t), g(t)\in D[t]$, suppose $f(t)g(t)=0$. Let $f(t) = \sum_{i=0}^n a_i t^i$, $g(t) = \sum_{j=0}^m b_jt^j$. ($a_i, b_j\in D$) If $f(t)$ and $g(t)$ are both non-zero 0, $a_nb_m\neq 0$. ($D$ is an integral domain) Thus by the definition of polynomial multiplication, $f(t)g(t)$ cannot be zero, since the leading coefficient is non-zero. Thus $f(t) = 0$ or $g(t)=0$.
\end{enumerate}~
\\
\prob{1.3.20.}
\begin{enumerate}
	\item If $a\neq 0$ and $b\neq 0$, $ab\neq 0$ since $D$ is an integral domain. This contradicts that $ab = 0$. Thus $a = 0$ or $b = 0$.
	\item $x = \pm 1$. But are these elements of $D$? Yes! $D$ is a ring with 1, and since $D$ is an abelian group, an inverse of 1 exists.
	\item $x^2-3x+2 = (x-2)(x+1)$. Thus $x = 2$ or $x = -1$. If $\ch(D) = 2$ or 3, other ``numbers" could be also possible (consider $\F_2$, $\F_3$) but they are actually $2\cdot 1_D$ and $(-1)\cdot 1_D$. Thus the roots are intrinsically the same.
\end{enumerate}~
\\
\prob{1.3.21.} No. If $6\cdot 1_D = 0$, $(2\cdot 1_D)\cdot (3\cdot 1_D) = 0$, and since $D$ is an integral domain, either $2\cdot 1$ or $3\cdot 1_D$ is 0. Thus contradicts that $\ch(D) = 6$.\\
\\
\obs{1.3.23.} Finite integral domain is a field.\\
\pf. Let $D = \{a_1, \dots, a_n\}$ and if $0\neq b\in D$, show that $b\in D\cross$. This holds by pigeonhole principle.\\
\\
\prob{1.3.24.} $R\times S = \{(r, s) \mid r\in R, s\in S\}$.
\begin{enumerate} 
	\item Define addition and multiplication on $R\times S$ as the following. If $r_i \in R$, $s_i\in S$,
	$$(r_1, s_1) + (r_2, s_2) = (r_1+r_2, s_1+s_2) \qquad (r_1, s_1)\cdot (r_2, s_2) = (r_1r_2, s_1s_2)$$
	It is easy to check that the ring axioms hold.
	\item Yes. (Check!)
	\item $(1, 1)$ is the multiplicative identity in $R\times S$. Yes.
	\item No. Consider $(r_1, 0)\cdot (0, s_1) = (0, 0) = 0$, where $r_1$ and $s_1$ are non-zero.
	\item No. $(1, 0)$ is a non-zero element, but does not have an inverse in $R\times S$.
\end{enumerate}

\pagebreak