\subsection{Elementary Properties of Rings}
\prob{1.2.2.} Use associativity and commutativity.
\begin{enumerate}
	\item $-a-b + a + b = -a +(-b) + a + b = -a + a + b + (-b) = (-a+a) + (b+(-b)) = 0 + 0 = 0$. Therefore $-a-b$ is the inverse of $a+b$. Thus $-(a+b) = -a-b$.
	\item $a + (-a) = 0$. The inverse of $-a$ is $a$. Thus $-(-a) = a$.
	\item (Induction) For $n = 0, 1$, trivial. For $n\geq 1$, suppose $-(na) = (-n)a$, which means that $(-n)a + na = 0$. Then $(n+1)a + (-(n+1))a = na + a + (-n)a + (-a)$, by associativity. Since $A$ is abelian, this is equal to $na + (-n)a + a + (-a) = 0 + 0 = 0$. Thus $-(n+1)a = (-(n+1))a$. For $n < 0$, substitute $m = -n > 0$. Then we need to show that $-(-ma) = ma$. The equation follows directly from (2), and the given property holds for all $n\in \Z$.
	\item No, Consider an abelian group $(\Z_2, +)$.
	Then $2(1) = 1 + 1 = 0$, but $1\neq 0$.\footnote{Group of 2 elements is unique, up to isomorphism.} Also consider $(\Z_3, +)$. Then $3(1) = 1 + 1 + 1 = 0$, but $1 \neq 0$.
\end{enumerate}~
\\
\prob{1.2.3.} \textbf{(Additive Exponent Law)} (Induction on $n$)
\begin{enumerate}
	\item For $n = 0$, $ma + na = ma + 0 = ma = (m+0)a$. Suppose for $n\geq 0$, $ma+na = (m+n)a$. $ma+(n+1)a = ma+na+a = (m+n)a + a = (m+n+1)a$. Now, for $n \leq 0$, suppose $ma+na = (m+n)a$. $ma + (n-1)a = ma + na - a = (m+n)a - a = (m+n-1)a$. Thus the given statement is true for all $m, n\in \Z$.
	\item For $n = 0$, $na+nb = 0 + 0 = 0 = 0(a+b)$. Suppose for $n\geq 0$, the equation holds. Then $(n+1)a+(n+1)b = na + a + nb + b = n(a+b) + (a+b) = (n+1)(a+b)$. Now, for $n\leq 0$, suppose the equation holds. Then $(n-1)a+(n-1)b = na - a + nb - b = n(a+b)-(a+b) = (n-1)(a+b)$. Thus the given statement is true for all $n\in \Z$.
	\item For $n = 0$, $m(na) = m(0) = 0 = (0)a = (mn)a$. Suppose for $n\geq 0$ the equation holds. Then $m((n+1)a)=m(na + a) = m(na) + ma = (mn)a + ma = (mn+m)a = (m(n+1))a$. ($\because (1), (2)$) Now, for $n\leq 0$, suppose the equation holds. Then $m((n-1)a) = m(na - a) = m(na)-ma = (mn)a-ma = (mn-m)a = (m(n-1))a$. Thus the given statement is true for all $m, n\in \Z$.
\end{enumerate}
Any abelian group is a $\Z$-module.\\
\\
이제부터는 $R = (R, +, \gop)$ 은 항상 ring 을 뜻한다.\\
\\
\obs{1.2.4.} $a, b\in R$ 이면,
\begin{enumerate}
	\item $0a = 0 = a0$. (단, $0\in R$)
	\item $(-a)b=-(ab)=a(-b)$. 따라서, $(-ab)$ 의 표기가 가능하다.
	\item $(-a)\cdot(-b)=ab$.
\end{enumerate}~
\pf. (3) $(-a)\cdot(-b) = -(a(-b)) = -(-(ab)) = ab$.\\
\\
\prob{1.2.5.} $a, b, a_i, b_j\in R$
\begin{enumerate}
	\item (Induction) If $n = 0$, $(0a)b = 0 = 0(ab) = a(0b)$. Suppose for $n\geq 0$, the equation holds. $((n+1)a)b = (na + a)b = (na)b + ab = n(ab) + ab = (n+1)(ab)$, $(na)b + ab = a(nb) + ab = a(nb + b) = a((n+1)b)$. Now for $n\leq 0$, suppose the equation holds. $((n-1)a)b = (na - a)b = (na)b - ab = n(ab) - ab = (n-1)(ab)$, $(na)b - ab = a(nb)-ab = a(nb- b) = a((n-1)b)$. The equation holds for all $n\in \Z$.\footnote{Any ring can be considered as a $\Z$-algebra.}
	\item (Induction) If $n = 1$, $\left(\sum_{i=1}^{m}a_i\right) \cdot b_1 = \sum_{1\leq i\leq m}a_ib_1$. Suppose for $n\geq 1$, the equation holds. Then $\left(\sum_{i=1}^{m}a_i\right)\cdot \left(\sum_{j=1}^{n+1} b_j\right) =\left(\sum_{i=1}^{m}a_i\right)\cdot \left(\sum_{j=1}^{n} b_j + b_{n+1}\right) = \sum_{1\leq i\leq m}\sum_{1\leq j\leq n} a_ib_j + \sum_{1\leq i\leq m}a_i b_{n+1} = \sum_{1\leq i\leq m}\sum_{1\leq j\leq n+1} a_ib_j$. Thus the equation holds for all $m, n\in \N$.
\end{enumerate}~
\\
\defn{1.2.6.} Ring $R$ 이 다음 조건
$$ab=ba \quad (a, b\in R)$$
을 만족하면, $R$ 을 \textbf{commutative ring} 이라고 부른다. 같은 방법으로 \textbf{commutative} $R$-\textbf{algebra} 도 정의한다.\\
\\
\defn{1.2.7.} $R$ 이 \textbf{multiplicative identity} 1 을 가지면, $R$ 을 ring with the multiplicative identity 1, 또는 간단히 [\textbf{ring with} 1] 이라고 부른다. 더 간단히 $1\in R$ 으로도 나타낸다.\\
\\
\prob{1.2.8.} Suppose the multiplicative identity is not unique. Then there exists two different multiplicative identities $x, y$. ($x\neq y$) But $y = xy = x$, contradicting that they are not equal. Thus the multiplicative identity must be unique.\\
\\
Ring with 1 에서는 항상 $1\neq 0$ 이라고 가정한다.\\
\\
\prob{1.2.11.}
\begin{enumerate}
	\item $(1_R + \cdots + 1_R)a = 1_R\cdot a + \cdots + 1_R\cdot a = a + \cdots +a = na$
	\item $(-1_R)a = 1_R\cdot(-a) = -a = (-a)\cdot 1_R = a(-1_R)$
	\item $(-1_R-\cdots-1_R)a = (-1_R)a + \cdots + (-1_R)a = -a -\cdots -a = (-n)a$ and $-a -\cdots -a =a(-1_R) +\cdots + a(-1_R) = a(-1_R-\cdots-1_R)$
\end{enumerate}~
\\
\textbf{\sffamily Notation 1.2.12.} $1 = 1_R\in R$ 일 때, 위 연습문제 1.2.11 은 $3 = 3_R = 1_R + 1_R+1_R$ 와 같이 표기하여도 별로 혼동이 없음을 보여 주고 있다. 앞으로 그렇게 표기하기로 한다. 조심할 점은 $3\in \Z$ 는 그 어떤 경우에도 zero 일 수 없지만, $3\in R$ 은 zero 일 수도 있고, $-1=5=8\in R$ 일 수도 있다는 점이다. 실제로 $\Z_3$ 에서 그렇다.\\
\\
\defn{1.2.13.} $R$ 이 [ring with 1] 이고, $a\in R$ 일 때,
\begin{center}
	$ab=1=ba$ 인 $b\in R$ 이 존재
\end{center}
하면 $a$ 를 an \textbf{invertible element} 또는 a \textbf{unit} 이라고 부른다. 이때, $b=a\inv$ 로 표기한다. 그리고,
$$R\cross = \{a\in R \mid a \text{ is a unit}\}$$
으로 표기하고, $R\cross$ 를 [the \textbf{unit group} in $R$] 이라고 부른다.\\
\\
\prob{1.2.14.} Suppose the multiplicative inverse of $a$ is not unique. Then there exists two different inverses $x, y\in R$ ($x\neq y$). By definition, $ax = 1_R = ay \imp x(ax) = x(ay) \imp (xa)x = (xa)y \imp x = y$, contradiction. The inverse is unique, if it exists.\\
\\
\prob{1.2.15.} Multiplication is associative, and if $a, b\in R\cross$, $ab\in R\cross$ since $ab\cdot b\inv\cdot a\inv = 1 = b\inv\cdot a\inv\cdot ab$ and $b\inv\cdot a\inv \in R$. The binary operation is well defined. Also $1_R \in R\cross$, which works as an identity element, and for all $a\in R\cross$, $a\inv \in R\cross$ by definition. Thus $R\cross$ is a multiplicative group.\\
\\
\prob{1.2.16.}
\begin{enumerate}
	\item $1\cdot 1 = 1$, thus $1\in R\cross$.
	\item In $R$, $b\inv \cdot a\inv$ is the inverse of $ab$. $ab\in R\cross$.
	\item If $a\in R\cross$, $\exists\,a\inv\in R$, and since the inverse of $a\inv$ is $a\in R$, $a\inv \in R\cross$ and $(a\inv)\inv = a$.
	\item[(2')] Since $a_1, \dots, a_n\in R\cross$, $a_1\inv, \dots, a_n\inv$ exist in $R$. Because $a_n\inv\cdots a_1\inv$ is the inverse of $a_1\cdots a_n$, $a_1\cdots a_n\in R\cross$. 
\end{enumerate}~
\\
\textbf{\sffamily Question 1.2.18.}
\begin{enumerate}
	\item The \textbf{ring of Gaussian integers} 를 $$\Z[\rmbf{i}] = \{m+n \rmbf{i} \in \C \mid m, n\in \Z\}$$ 로 정의하면, $\Z[\rmbf{i}]$ 는 당연히 ring 이 된다. 이 때, $\Z[\rmbf{i}]\cross$ 는? \footnote{$\Z[\rmbf{i}]\cross = \{\pm 1, \pm\rmbf{i}\}$}
	\item 앞 항을 일반화 하여 $d\in \Z$, $\sqrt{d}\in \Z$ 일 때, $$\Z[\sqrt{d}] = \{m+n\sqrt{d} \in \C\}$$ 로 정의하면 $\Z[\sqrt{d}]$ 도 당연히 ring 이 된다. 이 때, $\Z[\sqrt{d}]\cross$ 는? \footnote{...?}
\end{enumerate}~
\\
\prob{1.2.19.} \textbf{(Pascal's Triangle)} ~
$$\begin{aligned}
{n\choose i} + {n\choose i + 1}  &= \frac{n!}{i!\cdot(n-i)!} + \frac{n!}{(i+1)!\cdot(n-i-1)!} = \frac{n!}{i!\cdot (n-i-1)!} \left\{\frac{1}{n-i} + \frac{1}{i+1}\right\}\\
&=\frac{n!}{i!\cdot(n-i-1)!}\cdot \frac{n+1}{(n-i)(i+1)} = \frac{(n+1)!}{(i+1)!\cdot (n-1)!} = {n + 1\choose i+1}
\end{aligned}$$
By definition, ${0\choose 0} = {1\choose 0} = {1\choose 1} = 1$, and if ${k\choose i} \in \N$ for $k < n, 0\leq i \leq k$, ${n\choose i+1}$ can be expressed as the sum of two natural numbers (strong induction).\\
\\
\prob{1.2.20.} \textbf{(Binomial Theorem)}
\begin{enumerate}
	\item When $n = 1$, the given statement is trivial.
	\item Suppose the given equation is true for $n\geq 1$. We have
	$$(a+b)^n = \sum_{i=0}^n {n\choose i} a^i b^{n-i}$$
	\item For the inductive step,
	$$
	\begin{aligned}
		(a+b)^{n+1} &= (a+b)(a+b)^n = (a+b)\sum_{i=0}^n {n\choose i} a^i b^{n-i} \\ &= \sum_{i=0}^n {n\choose i} a^{i+1} b^{n-i} + \sum_{i=0}^n {n\choose i} a^i b^{n-i+1} \\
		&= a^{n+1} + \sum_{i=0}^{n-1} {n\choose i} a^{i+1} b^{n-i} + \sum_{i=1}^n {n\choose i} a^i b^{n-i+1} + b^{n+1} \\
		&= a^{n+1} + \sum_{i=0}^{n-1} {n\choose i} a^{i+1} b^{n-i} + \sum_{i=0}^{n-1} {n\choose i+1} a^{i+1} b^{n-i} + b^{n+1} \\
		&= a^{n+1} + \sum_{i=0}^{n-1} \left\{{n\choose i} + {n\choose i+1}\right\} a^{i+1} b^{n-i}  + b^{n+1} \\
		&= a^{n+1} + \sum_{i=0}^{n-1} {n+1\choose i+1} a^{i+1} b^{n-i}  + b^{n+1} \\
		&= a^{n+1} + \sum_{i=1}^{n} {n+1\choose i} a^{i} b^{n-i + 1}  + b^{n+1} = \sum_{i=0}^{n+1} {n+1\choose i} a^i b^{(n+1)-i}
	\end{aligned}
	$$
\end{enumerate}
Note that this works because $R$ is a commutative ring, and thus the statement holds for all $n\in \N$.\\
\\
\prob{1.2.21.} Since we have the exponential/distributive law, and most importantly, the \textbf{commutative} law, we can carry out most of the calculation easily, just like we learned in middle school.

\pagebreak