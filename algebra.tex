\documentclass[12pt]{article}
\usepackage{kotex}
\usepackage{amsmath}
\usepackage{amsfonts}
\usepackage{amssymb}
\usepackage{amsthm}
\usepackage{mathtools}
\usepackage{extarrows}
\usepackage{amscd}
\usepackage{titlesec}
\usepackage{chngcntr}
\usepackage{tikz-cd} 
\usepackage{fancyhdr}
\usepackage[utf8]{inputenc}
\usepackage[a4paper]{geometry}
\geometry{
	top = 20mm,
	bottom = 20mm,
	left = 20mm,
	right = 20mm
}

\usepackage{hyperref}
\hypersetup{
	colorlinks=true,
	linkcolor=black
}

\usepackage{enumitem}
\setlist[enumerate,1]{label={(\arabic*)}}

\setcounter{tocdepth}{2}
\setcounter{section}{0}

\counterwithin*{footnote}{section}
% \pagenumbering{gobble}
\renewcommand{\baselinestretch}{1.3}

\newcommand{\ds}{\displaystyle}
\newcommand{\abs}[1]{\left|#1\right|}
\newcommand{\paren}[1]{\left( #1 \right)}
\newcommand{\ceil}[1]{\left\lceil #1 \right\rceil}
\newcommand{\floor}[1]{\left\lfloor #1 \right\rfloor}
\renewcommand{\span}[1]{\left\langle #1 \right\rangle}
\newcommand{\norm}[1]{\left\lVert #1 \right\rVert}
\newcommand{\mf}[1]{\mathfrak{#1}}
\newcommand{\rmbf}[1]{\mathrm{\mathbf{#1}}}
\newcommand{\mc}[1]{\mathcal{#1}}
\newcommand{\bb}[1]{\mathbb{#1}}
\newcommand{\inv}{^{-1}}
\newcommand{\cross}{^\times}
\newcommand{\trans}{^{\mathrm{\mathbf{t}}}}
\newcommand{\adj}{\text{*}}
\newcommand{\ra}{\rightarrow}
\newcommand{\imp}{\implies}
\newcommand{\impb}{\impliedby}
\newcommand{\bs}{\setminus}
\newcommand{\N}{\mathbb{N}}
\newcommand{\Z}{\mathbb{Z}}
\newcommand{\Q}{\mathbb{Q}}
\newcommand{\R}{\mathbb{R}}
\newcommand{\C}{\mathbb{C}}
\newcommand{\F}{\mathbb{F}}
\DeclareMathOperator{\im}{im}
\DeclareMathOperator{\rk}{rk}
\DeclareMathOperator{\tr}{tr}
\DeclareMathOperator{\aut}{Aut}
\DeclareMathOperator{\diag}{diag}
\DeclareMathOperator{\ch}{char}
\DeclareMathOperator{\ann}{ann}
\DeclareMathOperator{\Ann}{Ann}
\newcommand{\nsub}{\mathrel{\unlhd}}
\newcommand{\pnsub}{\mathrel{\lhd}}
\newcommand{\mimp}{$\implies$}
\newcommand{\mimpb}{$\impliedby$}
\newcommand{\miff}{$\iff$}
\newcommand{\gop}{\text{곱}}
\newcommand{\sang}{\text{상}}
\newcommand{\tor}{_{\text{tor}}}

\renewcommand{\headrulewidth}{0.8pt}

\renewcommand{\contentsname}{목차}

\newcommand{\defn}[1]{\textbf{\sffamily Definition #1}}
\newcommand{\rmk}{\textbf{Remark}}
\newcommand{\ex}[1]{\textbf{\sffamily Example #1}}
\newcommand{\thm}[1]{\textbf{\sffamily Theorem #1}}
\newcommand{\pf}{\textit{Proof}}
\newcommand{\prop}[1]{\textbf{\sffamily Proposition #1}}
\newcommand{\prob}[1]{\textbf{\sffamily Exercise #1}}
\newcommand{\obs}[1]{\textbf{\sffamily Observation #1}}

\pagestyle{fancy}
\fancyhf{}
\lhead{\sffamily}
\rhead{\sffamily \thepage}

\title{\Large ~\\~\\~\\~\\학부 대수학 강의 II\\~\\~\\ \Huge \bfseries 대수학 \\~\\~\\~\\~\\}
\author{\itshape Sungchan Yi~\\~\\~\\}
\date{January, 2020}
\begin{document}
\maketitle
\pagebreak
\tableofcontents
\pagebreak

\section{Algebraic Structures I}
\subsection{Algebraic Structure}
\textbf{대수학}은 \textbf{algebraic structure}(\textbf{대수적 구조})를 공부하는 학문이다. 대수적 구조란 어떤 집합에 몇 개의 연산 구조가 주어진 것을 뜻한다. 우리는 \textbf{associative} binary operation 만을 생각한다.\footnote{모든 binary operation 은 associative 라고 가정한다.}\\
\\
\defn{1.1.1.} 이항연산 $\ast$ 를 갖는 집합 $G$ 가 다음 조건들
\begin{enumerate}
	\item[\sffamily (G1)] [모든 $g\in G$ 에 대하여 $g\ast e=e\ast g = g$] 인 원소 $e\in G$ 가 존재.
	\item[\sffamily (G2)] 각 $g \in G$ 에 대하여 [$g\ast \tilde{g} = \tilde{g}\ast g = e$ 인 원소 $\tilde{g}\in G$ 가 존재].
\end{enumerate}
을 만족하면 $(G, \ast)$ 를 \textbf{group}(\textbf{군}) 이라고 한다.\\

어떤 추가 조건들이 주어져 있는가에 따라 대수적 구조의 이름이 달라진다. 만약 위 정의에서 이항연산을 가진 집합 $G$ 가
\begin{enumerate}
	\item 아무런 추가 조건도 갖지 않으면 $G$ 를 \textbf{semigroup} 이라 한다.
	\item {\sffamily (G1)} 만을 만족하면, $G$ 를 \textbf{monoid} 라 한다.
	\item Group $G$ 가 $\abs{G}<\infty$ 도 만족하면, $G$ 를 \textbf{finite group} 이라 한다.
	\item Group $G$ 가 [$g\ast h = h\ast g$ for all $g, h\in G$] 도 만족하면 $G$ 를 \textbf{commutative group} (또는 \textbf{abelian group}) 이라 한다.
\end{enumerate}~
\\
\prob{1.1.3.} Consider $\N \cup \{0\}$ with addition. Associativity holds trivially, and $e = 0$. But there is no inverse for elements in $\N$.\\
\\
\defn{1.1.4.} $R$ 과 $X$ 가 집합일 때, 함수 $R\times X \ra X$ 를 $X$-위의 $R$-\textbf{상수곱} (\textbf{scalar multiplication}) 이라 한다. $a\in R$ 과 $x\in X$ 의 상수곱은 $a\cdot x = ax$ 로 표기한다. 또, $R$ 의 원소는 \textbf{scalar} 라고 부른다.\\
\\
\defn{1.1.7.} (\textbf{\sffamily Ring}) 집합 $R$ 이 \textbf{덧셈}과 \textbf{곱셈}이라는 이름의 두 개의 이항연산을 갖고 있을 때 다음 조건
\begin{enumerate}
	\item[\sffamily (R1)] $(a+b)c = ac+bc$, $a(b+c)=ab+ac$ \quad ($a, b, c\in R$) \quad (\textbf{분배법칙}(\textbf{distributive law}))
	\item[\sffamily (R2)] $(R, +)$ 는 abelian group
\end{enumerate}
을 만족하면, $(R, +, \gop)$ 을 \textbf{ring}(\textbf{환}) 이라 한다.\\
\\
\prob{1.1.8.} \textbf{Typical examples of ring}
\begin{enumerate}
	\item $(\Z, +,\gop)$
	\item $(\R, +,\gop)$
	\item $(\mf{M}_{n, n}(\R), +, \gop)$
	\item $(\R[t], +, \gop)$
\end{enumerate}
For all 4 examples, each set is an abelian group under addition, and the distributive law holds.\\

Alway remember: \textbf{우리가 아는 것은 행렬 뿐이다. 어떤 수학적 object 를 만나더라도 우리는 행렬(벡터공간)부터 생각한다.}\\
\\
\defn{1.1.10.} (\textbf{\sffamily $R$-module}) $R = (R, +, \gop)$ 이 ring 이고, 집합 $M$ 이 이항연산 \textbf{덧셈}과 $R$-\textbf{상수곱}을 갖고 있다고 하자.\footnote{$M$ 의 덧셈과 $R$ 의 덧셈은 분명히 구별해야 한다.} 다음 조건
\begin{enumerate}
	\item[\sffamily (M1)] $r(x+y) = rx + ry$ \quad ($r\in R,\;\; x, y, \in M$)
	\item[\sffamily (M2)] $(r+s)x = rx+sx$ \quad ($r, s\in R, \;\; x\in M$)
	\item[\sffamily (M3)] $r(sx) = (rs)x$ \quad ($r, s\in R, \;\; x\in M$)
	\item[\sffamily (M4)] $(M, +)$ 는 abelian group
\end{enumerate}
을 만족하면, $(M, +, \sang)$ 을 $R$-\textbf{module} ($R$-\textbf{가군}) 이라 한다.\\
\\
\prob{1.1.11.} \textbf{Typical examples of module}
\begin{enumerate}
	\item $x = (x_1, \dots, x_n), y = (y_1, \dots, y_n) \in \R^n$, $r, s\in \R$.
	\begin{enumerate}
		\item $r(x+y) = r(x_1+y_1, \dots, x_n+y_n) = (rx_1+ry_1, \dots, rx_n+ry_n) = rx+ry$
		\item $(r+s)x = (rx_1+sx_1, \dots, rx_n+sx_n) = rx + sx $
		\item $r(sx) = r(sx_1, \dots, sx_n) = (rsx_1, \dots, rsx_n) = (rs)x$
		\item Vector space $\R^n$ is an abelian group under addition.
	\end{enumerate}
	\item The proof is identical to (1).
\end{enumerate}~
\\
\prob{1.1.12.} $\R$-vector space was defined on a field ($\R$), while $R$-module is defined on a ring. A vector space over a field is a module over that field.\\
\\
\defn{1.1.13.} (\textbf{$R$-algebra}) $R=(R, +, \gop)$ 이 ring 이고, 집합 $\mc{A}$ 가 이항연산 \textbf{덧셈}과 \textbf{곱셈}, 그리고 $R$-\textbf{상수곱}을 갖고 있다고 하자. 다음 조건
\begin{enumerate}
	\item[\sffamily (A1)] $(\mc{A}, +, \gop)$ 은 ring
	\item[\sffamily (A2)] $(\mc{A}, +, \sang)$ 은 $R$-module
	\item[\sffamily (A3)] $(ra)b = r(ab) = a(rb)$ \quad ($r\in R, \;\; a, b, \in\mc{A}$)
\end{enumerate}
을 만족하면, $(\mc{A}, +, \gop, \sang)$ 을 $R$-\textbf{algebra} ($R$-\textbf{대수}) 이라 한다.\\
\\
\prob{1.1.15.} \textbf{Typical examples of algebra}
\begin{enumerate}
	\item $\R[t]$ (the $\R$-\textbf{algebra of polynomials}, the \textbf{polynomial algebra} over $\R$)
	\begin{enumerate}
		\item \textit{Is $(\R[t], +, \gop)$ a ring?} Yes.
		\item \textit{Is $(\R[t], +, \sang)$ an $R$-module?} Yes.
		\item For $r\in \R$, $f(t), g(t)\in \R[t]$,
		$(rf(t))g(t) = r(f(t)g(t)) = f(t)(rg(t))$.
	\end{enumerate}
	\item $\mf{M}_{n, n}(\R)$ (the $\R$-\textbf{algebra of $(n\times n)$-matrices}, the \textbf{matrix algebra} over $\R$)
	\begin{enumerate}
		\item \textit{Is $(\mf{M}_{n, n}(\R), +, \gop)$ a ring?} Yes.
		\item \textit{Is $(\mf{M}_{n, n}(\R), +, \sang)$ an $R$-module?} Yes.
		\item For $r \in \R$, $A, B \in \mf{M}_{n, n}(\R)$, $(rA)B = r(AB) = A(rB)$.
	\end{enumerate}
\end{enumerate}~
\\
\prob{1.1.17.} It is sufficient to only check {\sffamily (R1)}. For $a, b, c\in A$, $(a+b)c = 0 = 0 + 0 = ac + bc$, $a(b+c) = 0 = 0 + 0 = ab + ac$.\\
\\
\prob{1.1.18.}
\begin{enumerate}
	\item $R$ is an $R$-module. For $r, s, x, y\in R$,
	\begin{enumerate}
		\item $r(x+y) = rx+ry$ ($R$ is a ring)
		\item $(r+s)x = rx+sx$ ($R$ is a ring)
		\item $r(sx) = (rs)x$ (associativity)
	\end{enumerate}
	\item No. For $r, s, x, y \in R$,
	\begin{enumerate}
		\item \textit{Is $(R, +, \gop)$ a ring?} Yes.
		\item \textit{Is $(R, +, \sang)$ an $R$-module?} Yes.
		\item But $(rx)y$ may not equal $x(ry)$, since commutativity might not hold.
	\end{enumerate}
\end{enumerate}~
\\
\prob{1.1.19.} Show that $\mf{L}(V, V)$ is an $\R$-algebra.
\begin{enumerate}
	\item $(\mf{L}(V, V), +, \gop)$ is a ring. For $L, M, N \in \mf{L}(V, V)$,
	\begin{enumerate}
		\item $(L+M)N = LN + MN$, $L(M + N) = LM + LN$ (evaluate at $v\in V$ to check)
		\item $(\mf{L}(V, V), +)$ is an abelian group under addition. (composition is associative)
	\end{enumerate}
	\item $(\mf{L}(V, V), +, \sang)$ is an $\R$-module. For $r, s\in \R$, $L, M\in \mf{L}(V, V)$,
	\begin{enumerate}
		\item For $v\in V$, $r(L + M)(v) = rL(v) + rM(v) = (rL + rM)(v)$, thus $r(L + M) = rL + rM$.
		\item For $v\in V$, $(r+s)L(v) = rL(v) + sL(v) = (rL + sL)(v)$, thus $(r+s)L = rL + sL$.
		\item For $v\in V$, $r(sL)(v) = rsL(v) = (rs)L(v)$, thus $r(sL) = (rs)L$.
	\end{enumerate}
	\item For $r\in \R$, $L, M \in \mf{L}(V, V)$, $(rL)M = r(LM) = L(rM)$ (evaluate at $v\in V$ to check) 
\end{enumerate}
우리는 $\mf{L}(V, V)$ 를 \textbf{endomorphism algebra} on $V$ 라고 부른다.\\
\\
\prob{1.1.20.}
\begin{enumerate}
	\item $\R^n$ is an $\mf{M}_{n, n}(\R)$-module
	\begin{enumerate}
		\item $\mf{M}_{n, n}(\R)$ is a ring, and $(\R^n, +)$ is an abelian group under addition.
		\item $A(X+Y) = AX+AY$ \quad ($A\in \mf{M}_{n, n}(\R)$, $X, Y\in \R^n$)
		\item $(A+B)X = AX+BX$ \quad ($A, B\in \mf{M}_{n, n}(\R)$, $X\in \R^n$)
		\item $A(BX) = (AB)X$ \quad ($A, B\in \mf{M}_{n, n}(\R), X\in \R^n$)
	\end{enumerate}
	\item $V$ is an $\mf{L}(V, V)$-module
	\begin{enumerate}
		\item $\mf{L}(V, V)$ is a ring, and $V$ is an abelian group under addition. (vector space)
		\item $L(v+w) = L(v)+L(w)$ \quad ($L\in \mf{L}(V, V)$, $v, w\in V$)
		\item $(L + M)(v) = L(v) + M(v)$ \quad ($L, M\in \mf{L}(V, V)$, $v\in V$)
		\item $L(M(v)) = (LM)(v)$ \quad ($L, M \in \mf{L}(V, V)$, $v\in V$)
	\end{enumerate}
\end{enumerate}
We say that $V$ is an $\mf{L}(V, V)$-module with respect to the \textbf{natural action} of $\mf{L}(V, V)$. In short, $\mf{L}(V, V)$ \textbf{acts naturally} on $V$.\\
\\
\prob{1.1.21.} First, we know that $F[t]$ is a ring under multiplication, and $V$ is an abelian group under addition. For $v, w\in V$, $f(t), g(t)\in F[t]$,
\begin{enumerate}
	\item $f(t)\cdot (v+w) = f(T)(v+w) = f(T)(v)+f(T)(w) = f(t)\cdot v + f(t)\cdot w$
	\item $(f(t)+g(t))\cdot v = (f(T) + g(T))(v) = f(T)(v) + g(T)(v) = f(t)\cdot v + g(t)\cdot v$
	\item $(f(t)g(t))\cdot v = (f(T)g(T))(v) = f(T)(g(T)(v)) = f(T)(g(t)\cdot v) = f(t)\cdot (g(t)\cdot v)$
\end{enumerate}
Thus $V$ is an $F[t]$-module.\\
\\
\prob{1.1.22.} No, $\R$ is not an $\R[t]$-module. For $\alpha, \beta\in \R$ and $f(t)\in \R[t]$, $f(t)\cdot(\alpha + \beta) = f(\alpha + \beta)$, and $f(t)\cdot \alpha + f(t)\cdot \beta = f(\alpha) + f(\beta)$.
But generally, $f(\alpha + \beta) \neq f(\alpha) + f(\beta)$. The difference between the previous exercise is that $f(T) : V\ra V$ is a \textit{linear} operator, whereas (function) $f: \R \ra \R$ is non-linear in general.\\

\subsection{Elementary Properties of Rings}
\prob{1.2.2.} Use associativity and commutativity.
\begin{enumerate}
	\item $-a-b + a + b = -a +(-b) + a + b = -a + a + b + (-b) = (-a+a) + (b+(-b)) = 0 + 0 = 0$. Therefore $-a-b$ is the inverse of $a+b$. Thus $-(a+b) = -a-b$.
	\item $a + (-a) = 0$. The inverse of $-a$ is $a$. Thus $-(-a) = a$.
	\item (Induction) For $n = 0, 1$, trivial. For $n\geq 1$, suppose $-(na) = (-n)a$, which means that $(-n)a + na = 0$. Then $(n+1)a + (-(n+1))a = na + a + (-n)a + (-a)$, by associativity. Since $A$ is abelian, this is equal to $na + (-n)a + a + (-a) = 0 + 0 = 0$. Thus $-(n+1)a = (-(n+1))a$. For $n < 0$, substitute $m = -n > 0$. Then we need to show that $-(-ma) = ma$. The equation follows directly from (2), and the given property holds for all $n\in \Z$.
	\item No, Consider an abelian group $(\Z_2, +)$.
	Then $2(1) = 1 + 1 = 0$, but $1\neq 0$.\footnote{Group of 2 elements is unique, up to isomorphism.} Also consider $(\Z_3, +)$. Then $3(1) = 1 + 1 + 1 = 0$, but $1 \neq 0$.
\end{enumerate}~
\\
\prob{1.2.3.} \textbf{(Additive Exponent Law)} (Induction on $n$)
\begin{enumerate}
	\item For $n = 0$, $ma + na = ma + 0 = ma = (m+0)a$. Suppose for $n\geq 0$, $ma+na = (m+n)a$. $ma+(n+1)a = ma+na+a = (m+n)a + a = (m+n+1)a$. Now, for $n \leq 0$, suppose $ma+na = (m+n)a$. $ma + (n-1)a = ma + na - a = (m+n)a - a = (m+n-1)a$. Thus the given statement is true for all $m, n\in \Z$.
	\item For $n = 0$, $na+nb = 0 + 0 = 0 = 0(a+b)$. Suppose for $n\geq 0$, the equation holds. Then $(n+1)a+(n+1)b = na + a + nb + b = n(a+b) + (a+b) = (n+1)(a+b)$. Now, for $n\leq 0$, suppose the equation holds. Then $(n-1)a+(n-1)b = na - a + nb - b = n(a+b)-(a+b) = (n-1)(a+b)$. Thus the given statement is true for all $n\in \Z$.
	\item For $n = 0$, $m(na) = m(0) = 0 = (0)a = (mn)a$. Suppose for $n\geq 0$ the equation holds. Then $m((n+1)a)=m(na + a) = m(na) + ma = (mn)a + ma = (mn+m)a = (m(n+1))a$. ($\because (1), (2)$) Now, for $n\leq 0$, suppose the equation holds. Then $m((n-1)a) = m(na - a) = m(na)-ma = (mn)a-ma = (mn-m)a = (m(n-1))a$. Thus the given statement is true for all $m, n\in \Z$.
\end{enumerate}
Any abelian group is a $\Z$-module.\\
\\
이제부터는 $R = (R, +, \gop)$ 은 항상 ring 을 뜻한다.\\
\\
\obs{1.2.4.} $a, b\in R$ 이면,
\begin{enumerate}
	\item $0a = 0 = a0$. (단, $0\in R$)
	\item $(-a)b=-(ab)=a(-b)$. 따라서, $(-ab)$ 의 표기가 가능하다.
	\item $(-a)\cdot(-b)=ab$.
\end{enumerate}~
\pf. (3) $(-a)\cdot(-b) = -(a(-b)) = -(-(ab)) = ab$.\\
\\
\prob{1.2.5.} $a, b, a_i, b_j\in R$
\begin{enumerate}
	\item (Induction) If $n = 0$, $(0a)b = 0 = 0(ab) = a(0b)$. Suppose for $n\geq 0$, the equation holds. $((n+1)a)b = (na + a)b = (na)b + ab = n(ab) + ab = (n+1)(ab)$, $(na)b + ab = a(nb) + ab = a(nb + b) = a((n+1)b)$. Now for $n\leq 0$, suppose the equation holds. $((n-1)a)b = (na - a)b = (na)b - ab = n(ab) - ab = (n-1)(ab)$, $(na)b - ab = a(nb)-ab = a(nb- b) = a((n-1)b)$. The equation holds for all $n\in \Z$.\footnote{Any ring can be considered as a $\Z$-algebra.}
	\item (Induction) If $n = 1$, $\left(\sum_{i=1}^{m}a_i\right) \cdot b_1 = \sum_{1\leq i\leq m}a_ib_1$. Suppose for $n\geq 1$, the equation holds. Then $\left(\sum_{i=1}^{m}a_i\right)\cdot \left(\sum_{j=1}^{n+1} b_j\right) =\left(\sum_{i=1}^{m}a_i\right)\cdot \left(\sum_{j=1}^{n} b_j + b_{n+1}\right) = \sum_{1\leq i\leq m}\sum_{1\leq j\leq n} a_ib_j + \sum_{1\leq i\leq m}a_i b_{n+1} = \sum_{1\leq i\leq m}\sum_{1\leq j\leq n+1} a_ib_j$. Thus the equation holds for all $m, n\in \N$.
\end{enumerate}~
\\
\defn{1.2.6.} Ring $R$ 이 다음 조건
$$ab=ba \quad (a, b\in R)$$
을 만족하면, $R$ 을 \textbf{commutative ring} 이라고 부른다. 같은 방법으로 \textbf{commutative} $R$-\textbf{algebra} 도 정의한다.\\
\\
\defn{1.2.7.} $R$ 이 \textbf{multiplicative identity} 1 을 가지면, $R$ 을 ring with the multiplicative identity 1, 또는 간단히 [\textbf{ring with} 1] 이라고 부른다. 더 간단히 $1\in R$ 으로도 나타낸다.\\
\\
\prob{1.2.8.} Suppose the multiplicative identity is not unique. Then there exists two different multiplicative identities $x, y$. ($x\neq y$) But $y = xy = x$, contradicting that they are not equal. Thus the multiplicative identity must be unique.\\
\\
Ring with 1 에서는 항상 $1\neq 0$ 이라고 가정한다.\\
\\
\prob{1.2.11.}
\begin{enumerate}
	\item $(1_R + \cdots + 1_R)a = 1_R\cdot a + \cdots + 1_R\cdot a = a + \cdots +a = na$
	\item $(-1_R)a = 1_R\cdot(-a) = -a = (-a)\cdot 1_R = a(-1_R)$
	\item $(-1_R-\cdots-1_R)a = (-1_R)a + \cdots + (-1_R)a = -a -\cdots -a = (-n)a$ and $-a -\cdots -a =a(-1_R) +\cdots + a(-1_R) = a(-1_R-\cdots-1_R)$
\end{enumerate}~
\\
\textbf{\sffamily Notation 1.2.12.} $1 = 1_R\in R$ 일 때, 위 연습문제 1.2.11 은 $3 = 3_R = 1_R + 1_R+1_R$ 와 같이 표기하여도 별로 혼동이 없음을 보여 주고 있다. 앞으로 그렇게 표기하기로 한다. 조심할 점은 $3\in \Z$ 는 그 어떤 경우에도 zero 일 수 없지만, $3\in R$ 은 zero 일 수도 있고, $-1=5=8\in R$ 일 수도 있다는 점이다. 실제로 $\Z_3$ 에서 그렇다.\\
\\
\defn{1.2.13.} $R$ 이 [ring with 1] 이고, $a\in R$ 일 때,
\begin{center}
	$ab=1=ba$ 인 $b\in R$ 이 존재
\end{center}
하면 $a$ 를 an \textbf{invertible element} 또는 a \textbf{unit} 이라고 부른다. 이때, $b=a\inv$ 로 표기한다. 그리고,
$$R\cross = \{a\in R \mid a \text{ is a unit}\}$$
으로 표기하고, $R\cross$ 를 [the \textbf{unit group} in $R$] 이라고 부른다.\\
\\
\prob{1.2.14.} Suppose the multiplicative inverse of $a$ is not unique. Then there exists two different inverses $x, y\in R$ ($x\neq y$). By definition, $ax = 1_R = ay \imp x(ax) = x(ay) \imp (xa)x = (xa)y \imp x = y$, contradiction. The inverse is unique, if it exists.\\
\\
\prob{1.2.15.} Multiplication is associative, and if $a, b\in R\cross$, $ab\in R\cross$ since $ab\cdot b\inv\cdot a\inv = 1 = b\inv\cdot a\inv\cdot ab$ and $b\inv\cdot a\inv \in R$. The binary operation is well defined. Also $1_R \in R\cross$, which works as an identity element, and for all $a\in R\cross$, $a\inv \in R\cross$ by definition. Thus $R\cross$ is a multiplicative group.\\
\\
\prob{1.2.16.}
\begin{enumerate}
	\item $1\cdot 1 = 1$, thus $1\in R\cross$.
	\item In $R$, $b\inv \cdot a\inv$ is the inverse of $ab$. $ab\in R\cross$.
	\item If $a\in R\cross$, $\exists\,a\inv\in R$, and since the inverse of $a\inv$ is $a\in R$, $a\inv \in R\cross$ and $(a\inv)\inv = a$.
	\item[(2')] Since $a_1, \dots, a_n\in R\cross$, $a_1\inv, \dots, a_n\inv$ exist in $R$. Because $a_n\inv\cdots a_1\inv$ is the inverse of $a_1\cdots a_n$, $a_1\cdots a_n\in R\cross$. 
\end{enumerate}~
\\
\textbf{\sffamily Question 1.2.18.}
\begin{enumerate}
	\item The \textbf{ring of Gaussian integers} 를 $$\Z[\rmbf{i}] = \{m+n \rmbf{i} \in \C \mid m, n\in \Z\}$$ 로 정의하면, $\Z[\rmbf{i}]$ 는 당연히 ring 이 된다. 이 때, $\Z[\rmbf{i}]\cross$ 는? \footnote{$\Z[\rmbf{i}]\cross = \{\pm 1, \pm\rmbf{i}\}$}
	\item 앞 항을 일반화 하여 $d\in \Z$, $\sqrt{d}\in \Z$ 일 때, $$\Z[\sqrt{d}] = \{m+n\sqrt{d} \in \C\}$$ 로 정의하면 $\Z[\sqrt{d}]$ 도 당연히 ring 이 된다. 이 때, $\Z[\sqrt{d}]\cross$ 는? \footnote{...?}
\end{enumerate}~
\\
\prob{1.2.19.} \textbf{(Pascal's Triangle)} ~
$$\begin{aligned}
{n\choose i} + {n\choose i + 1}  &= \frac{n!}{i!\cdot(n-i)!} + \frac{n!}{(i+1)!\cdot(n-i-1)!} = \frac{n!}{i!\cdot (n-i-1)!} \left\{\frac{1}{n-i} + \frac{1}{i+1}\right\}\\
&=\frac{n!}{i!\cdot(n-i-1)!}\cdot \frac{n+1}{(n-i)(i+1)} = \frac{(n+1)!}{(i+1)!\cdot (n-1)!} = {n + 1\choose i+1}
\end{aligned}$$
By definition, ${0\choose 0} = {1\choose 0} = {1\choose 1} = 1$, and if ${k\choose i} \in \N$ for $k < n, 0\leq i \leq k$, ${n\choose i+1}$ can be expressed as the sum of two natural numbers (strong induction).\\
\\
\prob{1.2.20.} \textbf{(Binomial Theorem)}
\begin{enumerate}
	\item When $n = 1$, the given statement is trivial.
	\item Suppose the given equation is true for $n\geq 1$. We have
	$$(a+b)^n = \sum_{i=0}^n {n\choose i} a^i b^{n-i}$$
	\item For the inductive step,
	$$
	\begin{aligned}
		(a+b)^{n+1} &= (a+b)(a+b)^n = (a+b)\sum_{i=0}^n {n\choose i} a^i b^{n-i} \\ &= \sum_{i=0}^n {n\choose i} a^{i+1} b^{n-i} + \sum_{i=0}^n {n\choose i} a^i b^{n-i+1} \\
		&= a^{n+1} + \sum_{i=0}^{n-1} {n\choose i} a^{i+1} b^{n-i} + \sum_{i=1}^n {n\choose i} a^i b^{n-i+1} + b^{n+1} \\
		&= a^{n+1} + \sum_{i=0}^{n-1} {n\choose i} a^{i+1} b^{n-i} + \sum_{i=0}^{n-1} {n\choose i+1} a^{i+1} b^{n-i} + b^{n+1} \\
		&= a^{n+1} + \sum_{i=0}^{n-1} \left\{{n\choose i} + {n\choose i+1}\right\} a^{i+1} b^{n-i}  + b^{n+1} \\
		&= a^{n+1} + \sum_{i=0}^{n-1} {n+1\choose i+1} a^{i+1} b^{n-i}  + b^{n+1} \\
		&= a^{n+1} + \sum_{i=1}^{n} {n+1\choose i} a^{i} b^{n-i + 1}  + b^{n+1} = \sum_{i=0}^{n+1} {n+1\choose i} a^i b^{(n+1)-i}
	\end{aligned}
	$$
\end{enumerate}
Note that this works because $R$ is a commutative ring, and thus the statement holds for all $n\in \N$.\\
\\
\prob{1.2.21.} Since we have the exponential/distributive law, and most importantly, the \textbf{commutative} law, we can carry out most of the calculation easily, just like we learned in middle school.\\

\subsection{Fields and Integral Domains}
\defn{1.3.1.} $F$ 가 \textbf{commutative} [ring with 1] 일 때, $F$ 가 다음 조건
\begin{enumerate}
	\item[{\sffamily (F1)}] $F\cross = F - \{0\}$
\end{enumerate}
을 만족하면, $F$ 를 \textbf{field}(\textbf{체}) 라고 부른다. 이는 당연히
\begin{enumerate}
	\item[\sffamily (F1)'] $F$ 의 모든 \textbf{non-zero} element 는 invertible
\end{enumerate}
과 동치이다.\\
\\
\prob{1.3.2.} $F - \{0\}$ is commutative with respect to multiplication, and since $0$ always satisfies commutativity, $F$ is a commutative ring with 1. Now, from {\sffamily (f2)}, we know that every element in $F - \{0\}$ is invertible. (Every element in a group has an inverse) By {\sffamily (F1)'}, $F$ is a field.\\
\\
\prob{1.3.3.} \textbf{Typical examples of field}
\begin{enumerate} 
	\item $1\in \Q$, commutative, $\Q\cross = \Q - \{0\}$
	\item $1\in \R$, commutative, $\R\cross = \R - \{0\}$
	\item $1\in \C$, commutative, $\C\cross = \C - \{0\}$
	\item $1\in \Q(\sqrt{2})$, commutative, $\Q(\sqrt{2})\cross = \Q - \{0\}$
	\item $1\in \F_2$, commutative by definition, 0 is not invertible. $\F_2\cross = \F_2 - \{0\}$
\end{enumerate}~
\\
$\F_2$ is the \textbf{finite field} of order 2.\\
\\
\prob{1.3.4.} By the definition of additive/multiplicative identity, we can fill out the following Cayley table.
\begin{center}
	\begin{tabular}{c|cc}
		$+$ & 0 & 1 \\ \hline
		0 & 0 & 1\\
		1 & 1 & \\
	\end{tabular} \qquad
	\begin{tabular}{c|cc}
		$\times$ & 0 & 1 \\ \hline
		0 & & 0\\
		1 & 0 & 1\\
	\end{tabular}
\end{center}
We only need to define the result for $1+1$ and $0 \times 0$. If $1+1=1$, then the element 1 would not have an additive inverse, so $1+1$ should be 0. Also if $0\times 0 = 1$, then $0 = 0 \times 1 = 0\times(0+1) = 0\times 0 + 0\times 1 = 1 + 0 = 1$, contradicting $0\neq 1$. Thus $0\times 0 = 0$, and we can check that all the other properties hold.
\begin{center}
	\begin{tabular}{c|cc}
		$+$ & 0 & 1 \\ \hline
		0 & 0 & 1\\
		1 & 1 & 0\\
	\end{tabular} \qquad
	\begin{tabular}{c|cc}
		$\times$ & 0 & 1 \\ \hline
		0 & 0 & 0\\
		1 & 0 & 1\\
	\end{tabular}
\end{center}~
\\
\prob{1.3.5.} ...\\
\\
\defn{1.3.6.} Field $F$ 의 정의에서 $F$ 의 commutativity 조건을 제외하면, $F$ 를 \textbf{division ring} 이라고 부른다. 그리고, non-commutative division ring 은 \textbf{skew-field} 라고 부른다.\\
\\
\defn{1.3.8.} $a, b\neq 0$ 이 ring $R$ 의 원소일 때, 만약 $ab=0$ 이면, $a$ 와 $b$ 를 \textbf{zero divisor} 라고 부른다.\\
\\
\prob{1.3.9.}
\begin{enumerate}
	\item For any non-zero nilpotent element $a \in R$, suppose $m$ is the smallest natural number such that $a^m = 0$. Then $a, a^{m-1}\neq 0$ but $a^m = 0$, and $a$ is indeed a zero divisor.
	\item Suppose $a$ is a zero divisor. Then there exists two non-zero elements $a, b$ such that $ab = 0$. Since $a\in R\cross$, there exists $a\inv$. Multiplying $a\inv$ on the left gives $a\inv\cdot ab = (a\inv a)b = 1\cdot b = b = 0$, contradicting that $b$ is non-zero. Thus $a$ is not a zero divisor.
\end{enumerate}~
\\
\prob{1.3.10.}
\begin{enumerate}
	\item (\mimp) 1.3.9 (2)\\
	(\mimpb) Suppose $0\neq A\notin \mf{M}_{n, n}(\R)$. Then $A\inv$ does not exist. Then the columns of $A$ are linearly dependent, meaning that there exists $x\neq 0$ such that $Ax=0$. Consider another matrix $B$ where all its columns are $x$. Then $B\neq 0$, but $AB = 0$, contradicting that $A$ is not a zero divisor.
	\item When $R$ is finite, the answer is no. Suppose a non-zero element $a\in R$ exists, which is neither a zero divisor nor a unit. Consider a map $x\mapsto ax$ for all $x\in R$. If this map is injective, it has to be surjective. Then There exists $x\in R$ such that $ax = 1$, contradicting that $a$ is not a unit. If this map is not injective, there exists $u, v\in R$ ($u\neq v$) such that $au = av$. Then $a(u-v) = 0$, contradicting that $a$ is not a zero divisor. Thus a non-zero element is either a unit or a zero divisor.\\
	When $R$ is infinite, the answer is yes. Consider $2\in \Z$. $2$ is not a unit, and it is also not a zero divisor. 
\end{enumerate}~
\\
\prob{1.3.11.}
\begin{enumerate}
	\item Commutativity and finiteness is trivial from the operation table. And indeed, $a(b+c)=ab+ac$, $(a+b)c = ac+bc$ holds for all $a, b, c\in \Z_4$. Also, $\overline{1}$ is the multiplicative identity.
	\item $\overline{2}\cdot \overline{2} = \overline{0}$ but $\overline{2} \neq \overline{0}$. Thus $\overline{2}$ is a zero divisor.
	\item $4 = 1_R + 1_R + 1_R + 1_R = \overline{1} + \overline{1} + \overline{1} +\overline{1} = \overline{2} + \overline{1} + \overline{1} = \overline{3} + \overline{1} = \overline{0} = 0$.
	\item Multiplication is defined by $\overline{a}\cdot \overline{b}$ = (remainder of $ab$ divided by 4).
\end{enumerate}~
\\
\defn{1.3.13.} $1\in R$ 일 때, $0=n=n\cdot 1_R \in R$ 인 최소의 자연수 $n > 1$ 을 ring $R$ 의 \textbf{characteristic} 이라 부르고, $\ch(R) = n$ 으로 표기한다.\footnote{$n>1$ because if $n = 1$, $0=1$.} 만약 이러한 자연수 $n$ 이 존재하지 않으면, $R$ 은 \textbf{characteristic zero} 라고 하고, $\ch(R) = 0$ 으로 표기한다.\\
\\
\defn{1.3.14.} $1\in R$ 일 때, additive group $(R, +)$ 에서 $1_R$ 의 order $\abs{1_R}$ 이 finite 이면 $\ch(R) = \abs{1_R}$ 으로 정의한다. 한편, $\abs{1_R}=\infty$ 이면, $\ch(R)=0$ 으로 정의한다.\\
\\
\prob{1.3.15.}\\
(\mimp) For all $a\in R$, $na = (n\cdot 1_R)a =0\cdot a = 0$.\\
(\mimpb) Since $na = 0$ for all $a\in R$, $n\cdot 1_R = 0$.\\
\\
\prob{1.3.16.}\\
((1)\miff(2)) Holds by definition.\\
((2)\miff(3)) Since $a+a = 2a$, the result holds by 1.3.15.\\
((2)\miff(4)) Move the term $1_R$ to show the result.\\
\\
\prob{1.3.17.}
\begin{enumerate}
	\item For $\Z, \Q, \R, \C$, $n \cdot 1_R = n$ will never be 0, for all $n > 1$. Thus $\ch(\Z) =\ch(\Q) =\ch(\R) = \ch(\C) = 0$.
	\item Suppose $\ch(F) = n$, then $n\cdot 1_F = 0$. Since $1_{F[t]} = 1\in F$, we have $n\cdot 1_{F[t]} = 0$. Also because $1_{\mf{M}_{n, n}(F)} = \diag(1_F, \dots, 1_F) = I_n$, we have $n\cdot I_n = \diag(n\cdot 1_F, \dots, n\cdot 1_F) = 0$. Thus $\ch(F) = \ch(F[t]) = \ch(\mf{M}_{n, n}(F)) = n$. (For the proofs of $\ch(F[t]) = \ch(\mf{M}_{n, n}(F)) = n$, the minimality condition holds. If there exists $n > m\in \N - \{1\}$ s.t. $m\cdot 1_F = 0$, it contradicts $\ch(F) = n$.)
	\item $\overline{1} + \overline{1} = 2\cdot 1_{\bb{F}_2} = 0$, thus $\ch(\F_2) = 2$.
	\item From the operation table, it is obvious that $4\cdot \overline{1} = 0$, thus $\ch(\Z_4) = 4$.
\end{enumerate}~
\\
\defn{1.3.18.} \textbf{Commutative} ring with 1 $D$ 가 zero divisor 를 갖지 않으면, $D$ 를 \textbf{integral domain} 이라고 부른다.\\
\\
\prob{1.3.19.} \textbf{Typical examples of integral domain}
\begin{enumerate}
	\item $\Z$ is a commutative ring with 1. For $a, b\in \Z$, suppose $ab = 0$. If $a\neq 0$ and $b\neq 0$, $ab \neq 0$, contradicting that $ab=0$. Thus $a=0$ or $b=0$.
	\item $\Z[\bf{i}]$ is a commutative ring with 1. For $a, b\in \Z[\bf{i}]$, suppose $ab = 0$. Let $a = x_1+y_1\bf{i}$, $b = x_2+y_2\bf{i}$. ($x_i, y_i\in \Z$) Then $0 = ab = (x_1x_2-y_1y_2) + (x_1y_2+x_2y_1)\bf{i}$. Thus $x_1x_2 = y_1y_2$, $x_1y_2 + x_2y_1= 0$. Checking all possible cases gives $a = 0$ or $b=0$.
	\item A field is a commutative ring with 1. For $a, b\in F$, suppose $ab = 0$. If $a\neq 0$ and $b\neq 0$, there exists an inverse of $a$. Thus $a\inv \cdot a \cdot b = a\inv \cdot 0 = 0$, which gives $b=0$. This contradicts the assumption $b\neq 0$. Therefore $a = 0$ or $b = 0$.
	\item $D[t]$ is a commutative ring with 1. For $f(t), g(t)\in D[t]$, suppose $f(t)g(t)=0$. Let $f(t) = \sum_{i=0}^n a_i t^i$, $g(t) = \sum_{j=0}^m b_jt^j$. ($a_i, b_j\in D$) If $f(t)$ and $g(t)$ are both non-zero 0, $a_nb_m\neq 0$. ($D$ is an integral domain) Thus by the definition of polynomial multiplication, $f(t)g(t)$ cannot be zero, since the leading coefficient is non-zero. Thus $f(t) = 0$ or $g(t)=0$.
\end{enumerate}~
\\
\prob{1.3.20.}
\begin{enumerate}
	\item If $a\neq 0$ and $b\neq 0$, $ab\neq 0$ since $D$ is an integral domain. This contradicts that $ab = 0$. Thus $a = 0$ or $b = 0$.
	\item $x = \pm 1$. But are these elements of $D$? They may not be in $\F_2$.
	\item $x^2-3x+2 = (x-2)(x+1)$. Thus $x = 2$ or $x = -1$. If $\ch(D) = 2$ or 3, other answers can be also possible (consider $\F_2$, $\F_3$)
\end{enumerate}~
\\
\prob{1.3.21.} No. If $6\cdot 1_D = 0$, $(2\cdot 1_D)\cdot (3\cdot 1_D) = 0$, and since $D$ is an integral domain, either $2\cdot 1$ or $3\cdot 1_D$ is 0. Thus contradicts that $\ch(D) = 6$.\\
\\
\obs{1.3.23.} Finite integral domain is a field.\\
\pf. Let $D = \{a_1, \dots, a_n\}$ and if $0\neq b\in D$, show that $b\in D\cross$. This holds by pigeonhole principle.\\
\\
\prob{1.3.24.} $R\times S = \{(r, s) \mid r\in R, s\in S\}$.
\begin{enumerate} 
	\item Define addition and multiplication on $R\times S$ as the following. If $r_i \in R$, $s_i\in S$,
	$$(r_1, s_1) + (r_2, s_2) = (r_1+r_2, s_1+s_2) \qquad (r_1, s_1)\cdot (r_2, s_2) = (r_1r_2, s_1s_2)$$
	It is easy to check that the ring axioms hold.
	\item Yes. (Check!)
	\item $(1, 1)$ is the multiplicative identity in $R\times S$. Yes.
	\item No. Consider $(r_1, 0)\cdot (0, s_1) = (0, 0) = 0$, where $r_1$ and $s_1$ are non-zero.
	\item No. $(1, 0)$ is a non-zero element, but does not have an inverse in $R\times S$.
\end{enumerate}~\\

\subsection{Elementary Properties of Modules}
When we mention an $R$-module, we implicitly think that $1 \in R$. Moreover, $R$-module $M$ satisfies the condition

\begin{enumerate}
	\item[{\sffamily (M5)}] If $x\in M$, then $1x = x$.
\end{enumerate}~
\\
\defn{1.4.2.} If $F$ is a \textbf{division ring}, we call an $F$-module an $F$-\textbf{vector space}.\footnote{Note that we don't need commutativity in the definition of a module. So $F$ doesn't have to be a field.}\\
\\
\prob{1.4.3.}
\begin{enumerate}
	\item $0x = (0+0)x = 0x + 0x$. Thus $0x = 0$.
	\item $r0 = r(0+0) = r0 + r0$. Thus $r0 = 0\in M$.
	\item $(-1_R)x + x = (-1_R)x + 1_R\cdot x = (-1_R+1_R)x = 0x = 0$. Thus $-x = (-1_R)x$.
	\item $(-r)x + rx = (-r+r)x = 0x = 0$. Thus $-(rx) = (-r)x$.
	\item $rx + r(-x) = r(x + (-x)) = r0 = 0$. Thus $r(-x) = -(rx)$.
\end{enumerate}~
\\
\defn{1.4.5.} Let $M$ be an $R$-module.
\begin{enumerate}
	\item For $x\in M$, if there exists a \textbf{non-zero} element $r\in R$ such that $rx = 0$, $x$ is called an $r$-\textbf{torsion element}. And we say that ``$r$ \textbf{kills} $x$" or ``$r$ \textbf{annihilates} $x$".
	\item We define the \textbf{torsion part} of $M$ as $$M\tor = \{x\in M \mid r_x\cdot x = 0 \text{ for some non-zero } r_x\in R\}$$
	\item If $M\tor = M$, $M$ is a \textbf{torsion $R$}-\textbf{module}. If $M\tor = \{0\}$, $M$ is a \textbf{torsion-free $R$}-\textbf{module}.
	\item If $N\subseteq M$, we define the \textbf{annihilator ideal} of $N$ as
	$$\ann(N) = \ann_R(N) = \{r\in R\mid rx = 0 \text{ for all } x\in N\}$$
	and if $S\subseteq R$, we define $\Ann(S)$ as
	$$\Ann(S) = \Ann_M(S) = \{x\in M \mid rx = 0 \text{ for all }r\in S\}$$
\end{enumerate}
\prob{1.4.6.}
\begin{enumerate}
	\item 
\end{enumerate}

\pagebreak~
\end{document}
